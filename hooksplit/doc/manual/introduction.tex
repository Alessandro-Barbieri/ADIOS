\chapter{Introduction}

\section{Goals}

%\leftskip=0pt
%\parindent=0pt
As computational power has increased dramatically with the increase in the number 
of processors, input/output (IO) performance has become one of the most significant 
bottlenecks in today's high-performance computing (HPC) applications. With this 
in mind, ORNL and the Georgia Institute of Technology's Center for Experimental 
Research in Computer Systems have teamed together to design the Adaptive I/O System 
(ADIOS) as a componentization of the IO layer, which is scalable, portable, and 
efficient on different clusters or supercomputer platforms. We are also providing 
easy-to-use, high-level application program interfaces (APIs) so that application 
scientists can easily adapt the ADIOS library and produce science without diving 
too deeply into computer configuration and skills. 

\section{What is ADIOS?}

{\color{color01} ADIOS is a state-of-the-art componentization of the IO system 
that has demonstrated impressive IO performance results on leadership class machines 
and clusters; sometimes showing an improvement of more than 1000 times over well 
known parallel file formats. }ADIOS is essentially an I/O componentization of different 
I/O transport methods. This feature allows flexibility for application scientists 
to adopt the best I/O method for different computer infrastructures with very little 
modification of their scientific applications. ADIOS has a suite of simple, easy-to-use 
APIs. Instead of being provided as the arguments of APIs, all the required metadata 
are stored in an external Extensible Markup Language (XML) configuration file, 
which is readable, editable, and portable for most machines. 

\section{The Basic ADIOS Group Concept}

The ADIOS ``group'' is a concept in which input variables are tagged according 
to the functionality of their respective output files. For example, a common scientific 
application has checkpoint files prefixed with restart and monitoring files prefixed 
with diagnostics. In the XML configuration file, the user can define two separate 
groups with tag names of adios-group as ``restart'' and ``diagnostic.'' Each group 
contains a set of variables and attributes that need to be written into their respective 
output files. Each group can choose to have different I/O transport methods, which 
can be optimal for their I/O patterns.

\section{Other Interesting Features of ADIOS}

ADIOS contains a new self-describing file format, BP. The BP file format was specifically 
designed to support delayed consistency, lightweight data characterization, and 
resilience. ADIOS also contains python scripts that allow users to easily write 
entire ``groups'' with the inclusion of one include statement inside their Fortran/C 
code. Another interesting feature of ADIOS is that it allows users to use multiple 
I/O methods for a single group. This is especially useful if users want to write 
data out to the file system, simultaneously capturing the metadata in a database 
method, and visualizing with a visualization method.

The read API enables reading arbitrary subarrays of variables in a BP file and 
thus variables written out from N processor can be read in on arbitrary number 
of processors. ADIOS also takes care of the endianness problem at converting to 
the reader's architecture automatically at reading time. Matlab reader is included 
in the release while the VisIt parallel interactive visualization software can 
read BP files too (from version 2.0). 

ADIOS is fully supported on Cray XT and IBM BlueGene/P computers as well as on 
Linux clusters and Mac OSX. 

%\section{Future ADIOS 2.0 Goals}
%
%One of the main goals for ADIOS 2.0 is to produce faster reads via indexing methods. 
%Another goal is to provide more advanced data types via XML in ADIOS so that it 
%will be compatible with F90/c/C++ structures/objects. 
%
%We will also work on the following advanced topics for ADIOS 2.0: 
%
%\begin{itemize}
%    \item A link to an external database for provenance recording.
%
%    \item Autonomics through a feedback mechanism from the file system 
%to optimize I/O performance. For instance, ADIOS can be adaptively changed from 
%a synchronous to an asynchronous method or can decide when to write restart to 
%improve I/O performance.
%
%    \item A staging area for data querying, analysis, and in situ visualization.
%\end{itemize}

\section {What's new since version 1.3.1}

With ADIOS 1.4.0, there are several changes and new functionalities. 
The four major changes are in the Read API:

\begin{itemize}
\item No groups at reading anymore. You get all variables in one list.
There are no \verb+adios_gopen+ / \verb+adios_gclose+ / \verb+adios_inq_group+ 
calls after opening the file.
\item No time dimension. A 3D variable written multiple times will be seen as 
a 3D variable which has multiple steps (and not as single 4D variable as in adios 1.3.1).
Read requests should provide the number of steps to be read at once separately from the
spatial dimensions.
\item Multiple reads should be "scheduled" and then one \verb+adios_perform_reads()+ 
will do all at once. 
\item Selections. Instead of providing bounding box (offset and count values
in each dimension) in the read request itself, a selection has to be created 
beforehand. Besides bounding boxes, also list of individual points are supported 
as well as selections of a specific block from a particular writing process.
\end{itemize}

Overall, a single old \verb+adios_read_var()+ becomes three calls, but $n$ reads over the same subdomain requires $1+n+1$ calls.  
All changes were made towards in situ applications, to support streaming, non-blocking, chunking reads. 
Old codes can use the old read API too, for reading files but new users are definitely encouraged to use the new read API, even if they personally find the old one simpler to use for reading data from a file. The new API allows applications to move to in situ (staged, or memory-to-memory) processing of simulation data when file-based offline processing or code coupling becomes severely limited.  

Other new things in ADIOS:
\begin{itemize}
\item New read API. Files and streams can be processed step-by-step (or files with multiple steps at once). Multiple read requests are served at once, which enables for superior performance with some methods. Support for non-blocking and for chunked reads in memory-limited applications or for interleaving computation with data movement, although no current methods provide performance advantages in this release.  
\item Fortran90 modules for write and read API. Syntax of ADIOS calls can be checked by the Fortran compiler.
\item Java and Numpy bindings available (they should be built separately).
\item Visualization schema support in the XML configuration. Meshes can be described using output variables and data variables can be assigned to meshes. This will allow for automatic visualization from ADIOS-BP files with rich metadata, or to convey the developer's intentions to other users about how to visualize the data. A manual on the schema is separate from this Users' Manual and can be downloaded from the same web page. 
\item \emph{Skel} I/O skeleton generator for automatic performance evaluation of different methods. The XML configuration, that describes the output of an application, is used to generate code that can be used to test out different methods and to choose the best. Skel is part of ADIOS but it's manual is separate from this Users' Manual and can be downloaded from the same web page. 
\end{itemize}

