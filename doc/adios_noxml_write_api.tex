\vspace{22pt}
%\section*{{\Large \textbf{4 ADIOS No-XML Write API }}}
\section{ADIOS No-XML Write API }

\vspace{24pt}
\leftskip=0pt
ADIOS provides an option of writing data without loading an XML configuration file. 
This set of APIs is designed to cater to output data , which is not definable from 
the start of the simulation; such as an adaptive code. Using the no-XML API allows 
users to change their IO setup at runtime in a dynamic fashion.  This section discusses 
the details of no-XML write API's and demonstrates how they can be used in a program. 
 \label{HToc182553355}

\vspace{24pt}
\subsection*{{\large 4.1 }{\large \textbf{No-XML Write API Description}}}

\vspace{10pt}
This section lists routines that are needed for ADIOS no-XML functionalities. These 
routines prepare ADIOS metadata construction, for example, setting up groups, variables, 
attributes and IO transport method, and hence must be called before any other ADIOS 
I/O operations, i.e., adios\_open, adios\_group\_size, adios\_write, adios\_close. 
A common practice of using no-XML API is to first initialize ADIOS by calling adios\_init\_noxml 
and call adios\_select\_method to allocate necessary buffer for ADIOS to achieve 
best performance. Subsequently, declare a group via adios\_declare\_group and then 
adios\_define\_var API needs to be repetitively called to define every variable 
for the group.  In the end, adios\_select\_method needs to be called to choose 
a specific transport method.

\vspace{10pt}
\textbf{adios\_init\_noxml---}initialize no-XML ADIOS

\vspace{10pt}
\textbf{adios\_allocate\_buffer---}specify ADIOS buffer allocation strategy and 
buffer size in MB

\vspace{10pt}
\textbf{adios\_declare\_group---}declare an ADIOS group 

\vspace{10pt}
\textbf{adios\_define\_var---}define an ADIOS variable for an ADIOS group.

\vspace{10pt}
\textbf{adios\_define\_attribute---}define an ADIOS attribute for an ADIOS group

\vspace{10pt}
\textbf{adios\_select\_method---}associate an ADIOS transport method, such as MPI, 
POSIX method with a particular ADIOS group. The transport methods that are supported 
can be found at chapter 6\label{HToc182553356}

\vspace{22pt}
\subsubsection*{{\large \textbf{4.1.1 adios\_init\_noxml}}}

\vspace{10pt}
As opposed adios\_init, adios\_init\_noxml initialize ADIOS without loading XML 
configuration file. Note that adios\_init\_noxml is required to be called only 
once and before any other ADIOS API. 

\vspace{10pt}
\leftskip=22pt
int adios\_init\_noxml ()

\vspace{10pt}
\leftskip=22pt
Input: 

\vspace{10pt}
\leftskip=45pt
None

\vspace{22pt}
\leftskip=22pt
Fortran example: 

\vspace{10pt}
\parindent=13pt
call adios\_init\_noxml\_ (ierr)\label{HToc182553357}

\vspace{10pt}
\subsubsection*{{\large \textbf{4.1.2 adios\_allocate\_buffer}}}

\vspace{10pt}
\leftskip=0pt
\parindent=0pt
The adios\_allocate\_buffer routine allocates memory buffer for ADIOS internal. 

\vspace{10pt}
\parindent=18pt
int adios\_allocate\_buffer (

\vspace{10pt}
\parindent=54pt
enum ADIOS\_BUFFER\_ALLOC\_WHEN adios\_buffer\_alloc\_when

\vspace{10pt}
\leftskip=22pt
\parindent=68pt
,uint64\_t buffer\_size)

\vspace{10pt}
\leftskip=22pt
\parindent=0pt
Input: 

\vspace{10pt}
\leftskip=40pt
adios\_buffer\_alloc\_when - indicates when ADIOS buffer should be allocated. The 
value can be either {\small ADIOS\_BUFFER\_ALLOC\_NOW                          
   or ADIOS\_BUFFER\_ALLOC\_LATER.  }Please see section 5.3 for more details on 
ADIOS buffer.

\vspace{10pt}
\leftskip=36pt
\parindent=4pt
buffer\_size - the size of ADIOS buffer in MB. 

\vspace{10pt}
\leftskip=22pt
\parindent=0pt
Fortran example: 

\vspace{10pt}
\leftskip=40pt
call adios\_allocate\_buffer (10, adios\_err)

\vspace{22pt}
Note that, as opposed to C API, the Fortran API doesn't have adios\_buffer\_alloc\_when 
argument as it supports {\small ADIOS\_BUFFER\_ALLOC\_NOW }only as of the latest 
ADIOS version.\label{HToc182553358}

\vspace{10pt}
\subsubsection*{{\large \textbf{4.1.3 adios\_declare\_group}}}

\vspace{10pt}
\leftskip=0pt
This API is used to declare a new ADIOS group. The concept of ADIOS group, variable, 
attribute is detailed in the next chapter.

\vspace{10pt}
int adios\_declare\_group (int64\_t * id

\vspace{10pt}
\parindent=172pt
, const char * name

\vspace{10pt}
, const char * time\_index

\vspace{10pt}
\parindent=345pt
, enum ADIOS\_FLAG stats

\vspace{10pt}
\parindent=172pt
);

\vspace{10pt}
\leftskip=22pt
\parindent=0pt
Input: 

\vspace{10pt}
\leftskip=36pt
name - string containing the annotation name of the group 

\vspace{10pt}
time\_index - string containing the name of time attribute. If there is no time 
attribute, a null string (``'') should be passed

\vspace{10pt}
stats - a flag indicating whether or not to generate ADIOS statistics during writing, 
such as min/max/standard deviation. The value of \textit{stats} can be either adios\_flag\_yes{\Large  
}or adios\_flag\_no. If stats is set to adios\_flag\_yes, ADIOS internal calculates 
and outputs statistics for each processor automatically. The downside of turning 
stats on is that it consumes more CPU and memory during writing

\vspace{10pt}
\leftskip=22pt
Output: 

\vspace{10pt}
\leftskip=36pt
id - pointer to the ADIOS group structure

\vspace{22pt}
\leftskip=22pt
Fortran example: 

\vspace{10pt}
\parindent=13pt
call adios\_declare\_group (m\_adios\_group, \texttt{"}restart\texttt{"}, \texttt{"}iter\texttt{"}, 
1, adios\_err)\label{HToc182553359}

\vspace{10pt}
\subsubsection*{{\large \textbf{4.1.4 adios\_define\_var}}}

\vspace{10pt}
\leftskip=0pt
\parindent=0pt
This API is used to declare an ADIOS variable for a particular group. 

\vspace{10pt}
\leftskip=22pt
int adios\_define\_var (int64\_t group\_id, const char * name

\vspace{10pt}
\parindent=122pt
,const char * path

\vspace{10pt}
\parindent=151pt
,int type

\vspace{10pt}
,const char * dimensions

\vspace{10pt}
\parindent=302pt
,const char * global\_dimensions

\vspace{10pt}
\parindent=151pt
,const char * local\_offsets

\vspace{10pt}
);

\vspace{10pt}
Input: 

\vspace{10pt}
\leftskip=40pt
\parindent=0pt
group\_id - pointer to the internal group structure (returned by adios\_declare\_group 
call)

\vspace{10pt}
name - string containing the annotation name of a variable

\vspace{10pt}
path - string containing the path of an variable 

\vspace{10pt}
type - variable type 

\vspace{10pt}
dimensions - string containing variable local dimension. If the variable is a scalar, 
null string (``'') is expected. See 5.2.5 and 5.2.6 for details on ADIOS dimensions.

\vspace{10pt}
global\_dimensions - string containing variable global dimension. If the variable 
is a scalar or local array, null string (``'') is expected.

\vspace{10pt}
local\_offsets - string containing variable local offset. If the variable is a 
scalar or local array, null string (``'') is expected.

\vspace{22pt}
\parindent=-18pt
Output :

\vspace{10pt}
\leftskip=0pt
\parindent=18pt
None

\vspace{22pt}
\leftskip=22pt
\parindent=0pt
Fortran example: 

\vspace{10pt}
\leftskip=40pt
\parindent=3pt
call adios\_define\_var (m\_adios\_group, \texttt{"}temperature\texttt{"} \&

\vspace{10pt}
\parindent=93pt
,\texttt{"}\texttt{"}, 6 \&

\vspace{10pt}
,\texttt{"}NX\texttt{"}, \texttt{"}G\texttt{"}, \texttt{"}O\texttt{"}, adios\_err)\label{HToc182553360}

\vspace{10pt}
\subsubsection*{{\large \textbf{4.1.5 adios\_define\_attribute}}}

\vspace{10pt}
\leftskip=0pt
\parindent=0pt
This API is used to declare an ADIOS attribute for a particular group. See section 
5.2.3 for more details on ADIOS attribute.

\vspace{10pt}
\leftskip=22pt
int adios\_define\_attribute (int64\_t group

\vspace{10pt}
\parindent=187pt
,const char * name

\vspace{10pt}
,const char * path

\vspace{10pt}
\parindent=374pt
,enum ADIOS\_DATATYPES type

\vspace{10pt}
\parindent=187pt
,const char * value

\vspace{10pt}
,const char * var

\vspace{10pt}
\parindent=374pt
);

\vspace{10pt}
\parindent=0pt
Input:

\vspace{10pt}
\leftskip=45pt
group - pointer to the internal group structure (returned by adios\_declare\_group 
call)

\vspace{10pt}
\leftskip=117pt
\parindent=-72pt
name - string containing the annotation name of an attribute

\vspace{10pt}
\leftskip=45pt
\parindent=0pt
path - string containing the path of an attribute

\vspace{10pt}
type  - type of an attribute

\vspace{10pt}
value - pointer to a memory buffer that contains the value of the attribute

\vspace{10pt}
var - name of the variable which contains the attribute value. This argument needs 
to be set if argument ``value'' is null.  

\vspace{22pt}
\parindent=-22pt
Output:

\vspace{10pt}
\parindent=-4pt
None

\vspace{22pt}
\leftskip=22pt
\parindent=0pt
Fortran example: 

\vspace{10pt}
\leftskip=0pt
\parindent=46pt
call adios\_define\_attribute (m\_adios\_group, \texttt{"}date\texttt{"} \&

\vspace{10pt}
\parindent=234pt
,\texttt{"}\texttt{"}, 9 \&

\vspace{10pt}
,\texttt{"}Feb 2010\texttt{"} , \texttt{"}\texttt{"} , adios\_err)\label{HToc182553361}

\vspace{10pt}
\subsubsection*{{\large \textbf{4.1.6 adios\_select\_method}}}

\vspace{10pt}
\parindent=0pt
This API is used to choose an ADIOS transport method for a particular group. 

\vspace{10pt}
\leftskip=103pt
\parindent=-81pt
int adios\_select\_method (int64\_t group, const char * method

\vspace{10pt}
\parindent=91pt
,const char * parameters

\vspace{10pt}
,const char * base\_path

\vspace{10pt}
\parindent=264pt
);

\vspace{10pt}
\leftskip=22pt
\parindent=0pt
Input:

\vspace{10pt}
group - pointer to the internal group structure (returned by adios\_declare\_group 
call)

\vspace{10pt}
method - string containing the name of transport method that will be invoked during 
ADIOS write. A list of currently supported ADIOS methods can be found at Chapter 
6.

\vspace{10pt}
parameters - string containing user defined parameters that are fed into transport 
method.  For example, in MPI\_AMR method, the number of subfiles to write can be 
set via this argument (section 6.1.5).  This argument will be ignored silently 
if a transport method doesn't support the given parameters.

\vspace{10pt}
base\_path -  string containing the root directory to use when writing to disk 
or similar purposes

\vspace{22pt}
\leftskip=22pt
Fortran example: 

\vspace{10pt}
call adios\_select\_method (m\_adios\_group, \texttt{"}MPI\texttt{"}, \texttt{"}\texttt{"}, 
\texttt{"}\texttt{"}, adios\_err)\label{HToc182553362}

\vspace{34pt}
\subsection*{{\large 4.2 Create a no-XML ADIOS program}}

\vspace{10pt}
\leftskip=0pt
Below is a programming example that illustrates how to write a double-precision 
array t and a double-precision array with size of NX using no-XML API.   A more 
advanced example on writing out data sub-blocks is listed in the appendix 14.3. 


\vspace{22pt}
program adios\_global

\vspace{10pt}
\parindent=14pt
implicit none

\vspace{10pt}
include 'mpif.h'

\vspace{10pt}
\parindent=28pt
character(len=256)      :: filename = \texttt{"}adios\_global\_no\_xml.bp\texttt{"}

\vspace{10pt}
\parindent=14pt
integer                 :: rank, size, i, ierr

\vspace{10pt}
integer,parameter       :: NX=10

\vspace{10pt}
\parindent=28pt
integer                 :: O, G

\vspace{10pt}
\parindent=14pt
real*8, dimension(NX)   :: t

\vspace{10pt}
integer                 :: comm

\vspace{22pt}
\parindent=28pt
integer                 :: adios\_err

\vspace{10pt}
\parindent=14pt
integer*8               :: adios\_groupsize, adios\_totalsize

\vspace{10pt}
integer*8               :: adios\_handle

\vspace{10pt}
\parindent=28pt
integer*8               :: m\_adios\_group

\vspace{22pt}
\parindent=14pt
call MPI\_Init (ierr)

\vspace{10pt}
call MPI\_Comm\_dup (MPI\_COMM\_WORLD, comm, ierr)

\vspace{10pt}
\parindent=28pt
call MPI\_Comm\_rank (comm, rank, ierr)

\vspace{10pt}
\parindent=14pt
call MPI\_Comm\_size (comm, size, ierr)

\vspace{22pt}
call adios\_init\_noxml (adios\_err)

\vspace{10pt}
\parindent=28pt
call adios\_allocate\_buffer (10, adios\_err)

\vspace{22pt}
\parindent=14pt
call adios\_declare\_group (m\_adios\_group, \texttt{"}restart\texttt{"}, \texttt{"}iter\texttt{"}, 
1, adios\_err)

\vspace{10pt}
call adios\_select\_method (m\_adios\_group, \texttt{"}MPI\texttt{"}, \texttt{"}\texttt{"}, 
\texttt{"}\texttt{"}, adios\_err)

\vspace{22pt}
! define a integer

\vspace{10pt}
\parindent=28pt
call adios\_define\_var (m\_adios\_group, \texttt{"}NX\texttt{"} \&

\vspace{10pt}
\parindent=93pt
,\texttt{"}\texttt{"}, 2 \&

\vspace{10pt}
,\texttt{"}\texttt{"}, \texttt{"}\texttt{"}, \texttt{"}\texttt{"}, adios\_err)

\vspace{10pt}
\parindent=108pt
! define a integer

\vspace{10pt}
\parindent=14pt
call adios\_define\_var (m\_adios\_group, \texttt{"}G\texttt{"} \&

\vspace{10pt}
\parindent=93pt
,\texttt{"}\texttt{"}, 2 \&

\vspace{10pt}
,\texttt{"}\texttt{"}, \texttt{"}\texttt{"}, \texttt{"}\texttt{"}, adios\_err)

\vspace{10pt}
\parindent=108pt
! define a integer

\vspace{10pt}
\parindent=14pt
call adios\_define\_var (m\_adios\_group, \texttt{"}O\texttt{"} \&

\vspace{10pt}
\parindent=93pt
,\texttt{"}\texttt{"}, 2 \&

\vspace{10pt}
,\texttt{"}\texttt{"}, \texttt{"}\texttt{"}, \texttt{"}\texttt{"}, adios\_err)

\vspace{10pt}
\parindent=108pt
! define a global array

\vspace{10pt}
\parindent=14pt
call adios\_define\_var (m\_adios\_group, \texttt{"}temperature\texttt{"} \&

\vspace{10pt}
\parindent=93pt
,\texttt{"}\texttt{"}, 6 \&

\vspace{10pt}
,\texttt{"}NX\texttt{"}, \texttt{"}G\texttt{"}, \texttt{"}O\texttt{"}, adios\_err)

\vspace{22pt}
\parindent=108pt
call adios\_open (adios\_handle, \texttt{"}restart\texttt{"}, filename, \texttt{"}w\texttt{"}, 
comm, adios\_err)

\vspace{22pt}
\parindent=14pt
adios\_groupsize = 4 + 4 + 4 + NX * 8

\vspace{10pt}
call adios\_group\_size (adios\_handle, adios\_groupsize, adios\_totalsize, adios\_err)

\vspace{22pt}
\parindent=28pt
G = NX * size

\vspace{10pt}
\parindent=14pt
O = NX * rank

\vspace{10pt}
do i = 1, NX

\vspace{10pt}
\parindent=43pt
t(i)  = rank * NX + i - 1

\vspace{10pt}
\parindent=14pt
enddo

\vspace{22pt}
call adios\_write (adios\_handle, \texttt{"}NX\texttt{"}, NX, adios\_err)

\vspace{10pt}
\parindent=28pt
call adios\_write (adios\_handle, \texttt{"}G\texttt{"}, G, adios\_err)

\vspace{10pt}
\parindent=14pt
call adios\_write (adios\_handle, \texttt{"}O\texttt{"}, O, adios\_err)

\vspace{10pt}
call adios\_write (adios\_handle, \texttt{"}temperature\texttt{"}, t, adios\_err)

\vspace{22pt}
\parindent=28pt
call adios\_close (adios\_handle, adios\_err)

\vspace{22pt}
\parindent=14pt
call MPI\_Barrier (comm, ierr)

\vspace{22pt}
call adios\_finalize (rank, adios\_err)

\vspace{22pt}
\parindent=28pt
call MPI\_Finalize (ierr)

\vspace{10pt}
\parindent=0pt
end program


\vspace{22pt}
\leftskip=18pt
{\color{color20} \textbf{Figure 2. ADIOS no-XML example\label{HToc182553363}}}
