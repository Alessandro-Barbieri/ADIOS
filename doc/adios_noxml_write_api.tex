%\chapter*{{\Large \textbf{4 ADIOS No-XML Write API }}}
\chapter{ADIOS No-XML Write API }

ADIOS provides an option of writing data without loading an XML configuration file. 
This set of APIs is designed to cater to output data , which is not definable from 
the start of the simulation; such as an adaptive code. Using the no-XML API allows 
users to change their IO setup at runtime in a dynamic fashion.  This section discusses 
the details of no-XML write API's and demonstrates how they can be used in a program. 
 \label{HToc182553355}

\section{No-XML Write API Description}

This section lists routines that are needed for ADIOS no-XML functionalities. These 
routines prepare ADIOS metadata construction, for example, setting up groups, variables, 
attributes and IO transport method, and hence must be called before any other ADIOS 
I/O operations, i.e., adios\_open, adios\_group\_size, adios\_write, adios\_close. 
A common practice of using no-XML API is to first initialize ADIOS by calling adios\_init\_noxml 
and call adios\_select\_method to allocate necessary buffer for ADIOS to achieve 
best performance. Subsequently, declare a group via adios\_declare\_group and then 
adios\_define\_var API needs to be repetitively called to define every variable 
for the group.  In the end, adios\_select\_method needs to be called to choose 
a specific transport method.

\textbf{adios\_init\_noxml---}initialize no-XML ADIOS

\textbf{adios\_allocate\_buffer---}specify ADIOS buffer allocation strategy and 
buffer size in MB

\textbf{adios\_declare\_group---}declare an ADIOS group 

\textbf{adios\_define\_var---}define an ADIOS variable for an ADIOS group.

\textbf{adios\_define\_attribute---}define an ADIOS attribute for an ADIOS group

\textbf{adios\_select\_method---}associate an ADIOS transport method, such as MPI, 
POSIX method with a particular ADIOS group. The transport methods that are supported 
can be found at chapter 6\label{HToc182553356}

\subsection{adios\_init\_noxml}

As opposed adios\_init, adios\_init\_noxml initialize ADIOS without loading XML 
configuration file. Note that adios\_init\_noxml is required to be called only 
once and before any other ADIOS API. 

\begin{lstlisting}[language=C,caption={},label={}]
int adios_init_noxml ()
\end{lstlisting}

Input: 
\begin{itemize}
\item None
\end{itemize}

Fortran example: 
\begin{lstlisting}[language=Fortran,caption={},label={}]
call adios_init_noxml_ (ierr)
\end{lstlisting}

\subsection{adios\_allocate\_buffer}

The adios\_allocate\_buffer routine allocates memory buffer for ADIOS internal. 
\begin{lstlisting}[language=C,caption={},label={}]
int adios_allocate_buffer (
	enum ADIOS_BUFFER_ALLOC_WHEN adios_buffer_alloc_when 
	,uint64_t buffer_size)
\end{lstlisting}

Input: 
\begin{itemize}
\item adios\_buffer\_alloc\_when - indicates when ADIOS buffer should be allocated. The value can be either {\small ADIOS\_BUFFER\_ALLOC\_NOW                          
   or ADIOS\_BUFFER\_ALLOC\_LATER.  }Please see section 5.3 for more details on ADIOS buffer.
\item buffer\_size - the size of ADIOS buffer in MB. 
\end{itemize}


Fortran example: 
\begin{lstlisting}[language=Fortran,caption={},label={}]
call adios_allocate_buffer (10, adios_err)
\end{lstlisting}

Note that, as opposed to C API, the Fortran API doesn't have adios\_buffer\_alloc\_when 
argument as it supports {\small ADIOS\_BUFFER\_ALLOC\_NOW }only as of the latest 
ADIOS version.\label{HToc182553358}

\subsection{adios\_declare\_group}

This API is used to declare a new ADIOS group. The concept of ADIOS group, variable, 
attribute is detailed in the next chapter.

\begin{lstlisting}[language=C,caption={},label={}]
int adios_define_var (int64_t group_id, const char * name
	,const char * path ,int type
	,const char * dimensions
	,const char * local_offsets );
\end{lstlisting}

Input: 

\begin{itemize}
\item name - string containing the annotation name of the group 

\item time\_index - string containing the name of time attribute. If there is no time 
attribute, a null string (``'') should be passed

\item stats - a flag indicating whether or not to generate ADIOS statistics during writing, 
such as min/max/standard deviation. The value of \textit{stats} can be either \item adios\_flag\_yes{\Large  
}or adios\_flag\_no. If stats is set to adios\_flag\_yes, ADIOS internal calculates 
and outputs statistics for each processor automatically. The downside of turning 
stats on is that it consumes more CPU and memory during writing
\end{itemize}

Output: 
\begin{itemize}
\item id - pointer to the ADIOS group structure
\end{itemize}

Fortran example: 
\begin{lstlisting}[language=Fortran,caption={},label={}]
call adios_declare_group (m_adios_group, "restart", "iter", 1, adios_err)
\end{lstlisting}

\subsection{adios\_define\_var}

This API is used to declare an ADIOS variable for a particular group. 
\begin{lstlisting}[language=C,caption={},label={}]
int adios_define_var (int64_t group_id, const char * name
	,const char * path 
	,int type
	,const char * dimensions
	,const char * global_dimensions
	,const char * local_offsets 
	);
\end{lstlisting}

Input: 
\begin{itemize}
\item group\_id - pointer to the internal group structure (returned by adios\_declare\_group 
call)

\item name - string containing the annotation name of a variable

\item path - string containing the path of an variable 

\item type - variable type 

\item dimensions - string containing variable local dimension. If the variable is a scalar, 
null string (``'') is expected. See 5.2.5 and 5.2.6 for details on ADIOS dimensions.

\item global\_dimensions - string containing variable global dimension. If the variable 
is a scalar or local array, null string (``'') is expected.

\item local\_offsets - string containing variable local offset. If the variable is a 
scalar or local array, null string (``'') is expected.
\end{itemize}

Output :
\begin{itemize}
\item None
\end{itemize}

Fortran example: 
\begin{lstlisting}[language=Fortran,caption={},label={}]
call adios_define_var (m_adios_group, "temperature" &
	,"", 6 &
	,"NX", "G", "O", adios_err)
\end{lstlisting}

\subsection{adios\_define\_attribute}

This API is used to declare an ADIOS attribute for a particular group. See section 
5.2.3 for more details on ADIOS attribute.

\begin{lstlisting}[language=C,caption={},label={}]
int adios_define_attribute (int64_t group
	,const char * name 
	,const char * path
	,enum ADIOS_DATATYPES type
	,const char * value 
	,const char * var
	);
\end{lstlisting}

Input:
\begin{itemize}
\item group - pointer to the internal group structure (returned by adios\_declare\_group 
call)

\item name - string containing the annotation name of an attribute

\item path - string containing the path of an attribute

\item type  - type of an attribute

\item value - pointer to a memory buffer that contains the value of the attribute

\item var - name of the variable which contains the attribute value. This argument needs 
to be set if argument ``value'' is null.  
\end{itemize}

Output:
\begin{itemize}
\item None
\end{itemize}

Fortran example: 
\begin{lstlisting}[language=Fortran,caption={},label={}]
call adios_define_attribute (m_adios_group, "date" &
	,"", 9 &
	,"Feb 2010" , "" , adios_err)
\end{lstlisting}

\subsection{adios\_select\_method}
This API is used to choose an ADIOS transport method for a particular group. 

\begin{lstlisting}[language=C,caption={},label={}]
int adios_select_method (int64_t group, const char * method
	,const char * parameters 
	,const char * base_path
	);
\end{lstlisting}

Input:
\begin{itemize}
\item group - pointer to the internal group structure (returned by adios\_declare\_group 
call)

\item method - string containing the name of transport method that will be invoked during 
ADIOS write. A list of currently supported ADIOS methods can be found at Chapter 
.

\item parameters - string containing user defined parameters that are fed into transport 
method.  For example, in MPI\_AMR method, the number of subfiles to write can be 
set via this argument (see section \ref{section-method-mpiamr}).  This argument will be ignored silently 
if a transport method doesn't support the given parameters.

\item base\_path -  string containing the root directory to use when writing to disk 
or similar purposes
\end{itemize}

Fortran example: 
\begin{lstlisting}[language=Fortran,caption={},label={}]
call adios_select_method (m_adios_group, "MPI", "", "", adios_err)
\end{lstlisting}

\section{Create a no-XML ADIOS program}

Below is a programming example that illustrates how to write a double-precision 
array t and a double-precision array with size of NX using no-XML API.   A more 
advanced example on writing out data sub-blocks is listed in the appendix 14.3. 

\begin{lstlisting}[language=Fortran,caption={ADIOS no-XML example},label={}]
program adios_global implicit none
	include "mpif.h"
	character(len=256) :: filename = "adios_global_no_xml.bp" 
	integer :: rank, size, i, ierr
	integer,parameter :: NX=10
	integer :: O, G
	real*8, dimension(NX) :: t integer :: comm
	integer :: adios_err
	integer*8 :: adios_groupsize, adios_totalsize integer*8 :: adios_handle
	integer*8 :: m_adios_group

	call MPI_Init (ierr)
	call MPI_Comm_dup (MPI_COMM_WORLD, comm, ierr)
	call MPI_Comm_rank (comm, rank, ierr) 
	call MPI_Comm_size (comm, size, ierr)
	call adios_init_noxml (adios_err)
	call adios_allocate_buffer (10, adios_err)
	call adios_declare_group (m_adios_group, "restart", "iter", 1, adios_err) 
	call adios_select_method (m_adios_group, "MPI", "", "", adios_err)
	
	! define a integer
	call adios_define_var (m_adios_group, "NX" & 
		,"", 2 &
		,"", "", "", adios_err) ! define a integer
	call adios_define_var (m_adios_group, "G" & 
		,"", 2 &
		,"", "", "", adios_err) ! define a integer
	call adios_define_var (m_adios_group, "O" & 
		,"", 2 &
		,"", "", "", adios_err)
		
	! define a global array
	call adios_define_var (m_adios_group, "temperature" & 
		,"", 6 &
		,"NX", "G", "O", adios_err)
	call adios_open (adios_handle, "restart", filename, "w", comm, adios_err)
	
	adios_groupsize = 4 + 4 + 4 + NX * 8
	call adios_group_size (adios_handle, adios_groupsize &
		,adios_totalsize, adios_err)
	G = NX * size 
	O = NX * rank 
	do i = 1, NX
		t(i) = rank * NX + i - 1 
	enddo
	
	call adios_write (adios_handle, "NX", NX, adios_err)
	call adios_write (adios_handle, "G", G, adios_err)
	call adios_write (adios_handle, "O", O, adios_err)
	call adios_write (adios_handle, "temperature", t, adios_err)
	call adios_close (adios_handle, adios_err) 
	call MPI_Barrier (comm, ierr)
	call adios_finalize (rank, adios_err)
	call MPI_Finalize (ierr) 
end program
\end{lstlisting}
