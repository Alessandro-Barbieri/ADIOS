\vspace{34pt}
%\section*{{\Large \textbf{10 Converters}}}
\section{Converters}

\vspace{24pt}
\leftskip=0pt
To make BP files compatible with the popular file formats, we provide a series 
of converters to convert BP files to HDF5, NETCDF, or ASCII. As long as users give 
the required schema via the configuration file, the different converter tools currently 
in ADIOS have the features to translate intermediate BP files to the expected HDF5, 
NetCDF, or ASCII formats.\label{HToc82067530}

\vspace{24pt}
\subsection*{{\large 10.1 }{\large \textbf{ \label{HToc84890282}\label{HToc212016658}\label{HToc212016900}\label{HToc182553431}bp2h5}}}

\vspace{10pt}
This converter, as indicated by its name, can convert BP files into HDF5 files. 
Therefore, the same postprocessing tools can be used to analyze or visualize the 
converted HDF5 files, which have the same data schema as the original ones. The 
converter can match the row-based or column-based memory layout for datasets inside 
the file based on which language the source codes are written in.  If the XML file 
specifies global-bounds information, the individual sub-blocks of the dataset from 
different process groups will be merged into one global the dataset in HDF file.\label{HToc82067531}

\vspace{10pt}
\subsection*{{\large 10.2 }{\large \textbf{ \label{HToc84890283}\label{HToc212016659}\label{HToc212016901}\label{HToc182553432}bp2ncd}}}

\vspace{10pt}
The bp2ncd converter is used to translate bp files into NetCDF files. In Chap. 
5, we describe the time-index as an attribute for adios-group. If the variable 
is time-based, one of its dimensions needs to be specified by this time-index variable, 
which is defined as an unlimited dimension in the file into which it is to be converted. 
a NetCDF dimension has a name and a length. If the constant value is declared as 
a dimension value, the dimension in NetCDF will be named varname\_n, in which varname 
is the name of the variable and n is the nth dimension for that variable. To make 
the name for the dimension value more meaningful, the users can also declare the 
dimension value as an attribute whose name can be picked up by the converter and 
used as the dimension name.

\vspace{10pt}
Based on the given global bounds information in a BP file, the converter can also 
reconstruct the individual pieces from each process group and create the global 
space array in NetCDF. A final word about editing the XML file: the name string 
can contain only letters, numbers or underscores (``\_''). Therefore, the attribute 
or variable name should conform to this rule. \label{HToc84890284}\label{HToc212016660}\label{HToc212016902}\label{HToc82067532}\label{HToc182553433}

\vspace{10pt}
\subsection*{{\large 10.3 }{\large \textbf{bp2ascii}}}

\vspace{10pt}
Sometimes, scientists want to extract one variable with all the time steps or want 
to extract several variables at the same time steps and store the resulting data 
in ASCII format. The Bp2ascii converter tool allows users to accomplish those tasks. 

\vspace{10pt}
Bp2ascii bp\_filename -v x1 ... xn [-c/-r] -t m,n

\vspace{10pt}
-v - specify the variables need to be printed out in ASCII file

\vspace{10pt}
-c -print variable values for all the time steps in column

\vspace{10pt}
-r - print variable values for all the time steps in row

\vspace{10pt}
-t - print variable values for time step m to n,  if not defined, all the time 
steps will be printed out.

\vspace{10pt}
\subsection*{{\large 10.4 }{\large \textbf{ \label{HToc84890285}\label{HToc212016661}\label{HToc212016903}\label{HToc182553434}Parallel 
Converter Tools}}}

\vspace{10pt}
Currently, all of the converters mentioned above can only sequentially parse bp 
files. We will work on developing parallel versions of all of the converters for 
improved performance. As a result, the extra conversion cost to translate bp into 
the expected file format can be unnoticeable compared with the file transfer time. 
 \label{HToc84890286}\label{HToc212016662}\label{HToc212016904}\label{HToc182553435}
