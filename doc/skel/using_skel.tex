\chapter{Using Skel}
\section{Overview}
Figure 1 shows the typical workflow of using skel to create a skeletal I/O 
benchmark. The example uses the GTS application, and thus the workflow begins
with gts.xml, the XML descriptor from GTS. The {\it skel xml} command is used
to create a second xml file, gts\_skel.xml which will serve as the ADIOS xml
descriptor for the skeletal application. Next, skel params is used to generate
a parameters file. The generated parameters file is then edited by the user to
guide the subsequent generation of the skeletal application.

Figure 1: Skel Workflow

At this point, the commands {\it skel source}, {\it skel makefile}, and {\it skel submit}
may be used to generate the source files, Makefile, and submission scripts that
comprise the skeletal application. With all of the components of the skeletal
application created, it is now time to build the application, using {\it make}, and
finally deploy the application to a directory from which it may be launched,
using {\it make deploy}, which copies the gts\_skel.xml file, executable files, and
submission script. The user now has a ready to run I/O benchmark without
having written any source code at all.


\section{Detailed Example}
In this section we will describe the steps used to create an I/O Kernel based
on the GTS application. The ADIOS config file for GTS can be found in the
examples directory.

Coming Soon
