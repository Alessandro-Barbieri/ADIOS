\chapter{Yaml File Format}

Skel supports various methods for specifying the high-level I/O description to be
used for creating skeletal applications. One of these is the yaml file, described below.
YAML is a data serialization language that is similar
to JSON. YAML supports data abstractions including sequences and mappings, and allows
these to be nested. General information about YAML is available at http://www.yaml.org/

The yaml format described here is the one that is produced by the {\tt skeldump} utility,
and which is accepted by the {\tt skel replay} command, both described elsewhere in
this manual.
\vspace{5mm}

At the top level of the yaml file, there are a series of mappings as follows:
\begin{description}
  \item[lang] specifies the target language, currently C and fortran are supported.

  \item[procs] indicates the number of MPI tasks involved in the I/O operations.

  \item[group] is the name of the ADIOS group in which to write the variables.

  \item[variables] is a sequence of mappings representing the variables to be written

\end{description}

\vspace{5mm}
Each variable is represented by a nested mapping that consists of:
\begin{description}

  \item[name] The name of the variable

  \item[type] The unit type of the variable, should correspond to a valid type in the target language

  \item[dims] Either {\tt scalar}, or a comma separated list of array dimensions

  \item[value] (For scalars) a string representing the value to be assigned to this variable in the skeletal application code.

\end{description}


