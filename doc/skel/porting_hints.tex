
\chapter{Hints for Porting Skel}
Skel has been developed and tested on only a small handful of platforms. While
we expect most functionality will be portable to a wider range of machines,
there are likely to be some issues arising when running skel on your system.
There are a few hints in the following sections that may help you to get started.
If further assistance is needed to get skel working, please contact lot@ornl.gov.

\section{Makefiles}
Assuming that they work properly, Makefiles generated by skel are quite conve-
nient, as the skel user need not think about how to compile the code, but can
simply type make. All of the details are taken care of by skel. We have tested
skel on only a few systems at this point, and thus it is quite possible that the
Makefile generated by skel may fail on some systems. Users familiar with make
with a need to adjust some aspect of the generated makefiles should investigate
/.skel/templates/Makefile.default.tpl. This template file is used by Skel to
generate Makefiles, and can be adjusted to the needs of the user. The template
syntax is simple, with {\tt \$\$VAR\_NAME\$\$} used to indicate template substitutions
to be made by skel makefile. Again, if you run into trouble with this, please
contact us as described above.

\section{Submission Scripts}
Similar to the Makefile generation, skel uses templates to generate submission
scripts for the generated applications. The submission templates are also located
in
/.skel/templates/, and are named submit <target>.tpl, where <target>
corresponds to the submit target defined in the user's settings file. To create a
new submit target , simply copy one of the existing template files, and rename
it with the desired submit target name. Then, adjust the submission syntax
so that the generated files work properly with the submission mechanism on
your system. Once again, if you run into trouble with this, please contact us as
described above.
