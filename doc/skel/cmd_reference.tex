\chapter{Skel Command Reference}

Most skel commands are of the form:
skel
subcommand project
\section{Available Subcommands}

\begin{description}
  \item[skel install] \hfill \\
Usage:
skel install
  \item[skel makefile] \hfill \\
Generates a Makefile fit for building and deploying the skeletal
application.
Requires <project>\_skel.xml and <project>\_params.xml
Usage:
skel makefile <project>
  \item[skel params] \hfill \\
Generates a parameters file that can be customized by the user.
Note that this command creates a file called <project> params.xml.default
so as to avoid overwriting a customized parameters file. This means that
the user should copy this file to <project> params.xml and edit before
proceeding with code generation.
Requires <project>\_skel.xml
Usage:
skel params <project>
  \item[skel source] \hfill \\
Generates a C or Fortran code that performs the I/O operations
described by the XML descriptor and the parameters file.
Requires <project>\_skel.xml and <project>\_params.xml
Usage:
skel source <project>
  \item[skel submit] \hfill \\
Generates a submission script for the skeletal application
Requires <project>\_skel.xml and <project>\_params.xml
Usage:
skel submit <project>
  \item[skel xml] \hfill \\
Generates the <project>\_skel.xml file.
Requires <project>.xml
Usage:
skel xml <project>
\end{description}
