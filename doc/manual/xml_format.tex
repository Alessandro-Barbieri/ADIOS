\chapter{XML Config File Format}
\label{chapter-xml}

\section{Overview} 
XML is designed to allow users to store as much metadata as they can in an external 
configuration file. Thus the scientific applications are less polluted and require 
less effort to be verified again.

First, we present the XML template. Second, we demonstrate how to construct the 
XML file from the user's own source code. Third, we note how to troubleshoot and 
debug the errors in the file.  

Abstracting metadata, data type, and dimensions from the source code into an XML 
file gives users more flexibility to annotate the arrays or variables and centralizes 
the description of all the data structures, which in return, allows I/O componentization 
for different implementation of transport methods. By cataloguing the data types 
externally, we have an additional documentation source as well as a way to easily 
validate the write calls compared with the read calls without having to decipher 
the data reorganization or selection code that may be interspersed with the write 
calls. It is useful that the XML name attributes are just strings. The only restrictions 
for their content are that if the item is to be used in a dataset dimension, it 
must not contain commas and must contain at least one non-numeric character. This 
is useful for incorporating expressions as various array dimensions elements. Listing~\ref{list-xml-conf-example} llustrates the corresponding XML configuration for the example we demonstrated 
in Figure 1. 

At a minimum, a configuration document must declare an adios-config element. It 
serves as a container for other elements; as such, it MUST be used as the root 
element. The expected children in any order would be adios-group, method, and buffer. 
The main elements of the xml file format are of the format 

\begin{lstlisting}
<element-name attr1=value1 attr2=value2 ...>
\end{lstlisting}

\begin{lstlisting}[language=XML, caption={Example XML configuration}, label=list-xml-conf-example]
<?xml version="1.0"?>
<adios-config> 
    <adios-group>
        <var ... />
        <attribute .../>
    </adios-group>
    <method ... />
    <buffer ... />
</adios-config>
\end{lstlisting}

\section{adios-config}
The adios-config element is the container for all ADIOS elements, and thus practically
all configuration file has one of these elements that contain everything. Multiple 
elements are allowed, however, there is no know use for that. 

The only attribute that adios-config has is the \verb+host-language+ attribute for 
language declaration. Fortran or C should be chosen, according to the source 
language that is going to use ADIOS. The only difference that it makes is for
the use of \verb+time-index+ in a group (see \verb+adios_group+ below) when multiple output 
steps are appended to the
same file. In Fortran, the time index, as dimension declaration, should be the last
dimension (the slowest one), while in C, it should be the first one (again, the slowest
dimension).

\begin{lstlisting}[language=XML]
<adios-config host-language="Fortran">
    ...
</adios-config>
\end{lstlisting}


\section{adios-group}
The adios-group element represents a container for a list of variables that share 
the common I/O pattern as stated in the basic concepts of ADIOS in the first chapter. 
In this case, the group domain division logically corresponds to the different 
functions of output in scientific applications, such as restart, diagnosis, and 
snapshot. Depending on the different applications, adios-group can occur as many 
times as is needed.

\subsection{Declaration}

The following example illustrates how to declare an adios group inside an XML file. 
First we start with adios-group as our tag name, which is case insensitive. It 
has an indispensable attribute called ``name,'' whose value is usually defined 
as a descriptive string indicating the function of the group. In this case, the 
string is called ``restart,'' because the files into which this group is written 
are used as checkpoints. The second attribute ``host-language'' indicates the language 
in which this group's I/O operations are written. The value of attribute ``coordination-communicator'' 
is used to coordinate the operations on a shared file accessed by multiple processes 
in the same communicator domain. ``Coordination-var'' provides the ability to use 
the user-defined variable, for example mype, rather than an MPI communicator for 
file coordination. 

\begin{lstlisting}[language=XML]
<adios-group name="restart"
             host-language="C"
             coordination-communicator="comm"
             coordination-var="mype"
             time-index="iter"/>
\end{lstlisting}

Required:
\begin{itemize}
\item name---containing a descriptive string to name the group
\end{itemize}

Optional:
\begin{itemize}
\item host-language---language in which the source code for group is written
\item coordination-communicator---MPI-IO writing to a shared file
\item coordination-var---coordination variables for non-MPI methods, such as Datatap method
\item time-index---time attribute variable
\end{itemize}

\subsection{Variables}
\label{section-xml-variables}
The nested variable element ``var'' for adios\_group, which can be either an array 
or a primitive data type, is determined by the dimension attribute provided. 

\subsubsection{Declaration}

The following is an example showing how to define a variable in the XML file. 
\begin{lstlisting}[language=XML]
<var name="z-plane ion particles"
     gwrite="zion" 
     gread="zion_read" 
     type="adios_real" 
     dimensions="7,mimax" 
     read="yes"/>
\end{lstlisting}

\subsubsection{Attribute list}
The attributes associated with var element  as follows: 

Required:
\begin{itemize}
\item name - the string name of variable stored in the output file
\item type - the data type of the variable
\end{itemize}

Optional: 
\begin{itemize}
\item gwrite - the value will be used in the \verb+gpp.py+ python script as the 
variable name in the source code to generate adios\_write 
routines; the default value is the value of the attribute \textit{name} if gwrite 
is not defined. Use it if the write code is automatically generated and the desired
variable name in the output is different from the name that contains the data in the
program. The value is substituted 'as is' into the generated code, so arbitrary
Fortran/C expressions can be used. 
\item gread - the value will be used in the python scripts to generate adios\_read routines' 
the default value is the value of the attribute \textit{name} if  gread is not 
defined.
\item path - HDF-5-style path for the element or path to the HDF-5 group or data item 
to which this attribute is attached.  The default value is ``/''.
\item dimensions - a comma-separated list of numbers and/or names that correspond to 
integer var elements determine the size of this item. If not specified, the variable 
is scalar.
\item read - value is either \textit{yes} or \textit{no}; in the case of no, the adios\_read 
routine will not be generated for this var entry. If undefined, the default value 
will be yes. 
\end{itemize}

\subsection{Attributes}
The attribute element for adios\_group provides the users with the ability to specify 
more descriptive information about the variables or group. The attributes can be 
defined in both static or dynamic fashions. ADIOS supports only scalar attributes, i.e, 
no arrays or vectors are supported. Note that a string is considered a scalar in ADIOS. 
From ADIOS 1.4, only the root process of the application writes the attribute into 
the output, therefore, process-dependent information cannot be save as an attribute 
(by definition, that is a variable information, which should be stored in a variable).

\subsubsection{Declaration}

The static type of attributes can be defined as follows:
\begin{lstlisting}[language=XML]
<attribute name="experimental date"
    path="/zion"
    value="Sep-19-2008" 
    type="adios_real"/>
\end{lstlisting}

If an attribute has dynamic value that is determined by the runtime execution of 
the program, it can be specified as follows:
\begin{lstlisting}[language=XML]
<attribute name="experimental date"
    path="/zion"
    var="time"/>
\end{lstlisting}
where var ``time'' need to be defined in the same adios-group.

\subsubsection{Attribute list}

Required:
\begin{itemize}
\item name -  name of the attribute
\item path - hierarchical path inside the file for the attribute
\item value - attribute has static value of the attribute, mutually exclusive with the  attribute \textit{var}
\item type - string or numeric type, paired with attribute \textit{value}, in other words, mutually exclusive with the attribute \textit{var} also
\item var - attribute has dynamic value that is defined by a variable in \textit{var}
\end{itemize}

\subsection{Gwrite src}

The element \texttt{<}Gwrite src=\texttt{"}...\texttt{"}\texttt{>} is unlike \texttt{<}var\texttt{>} 
or \texttt{<}attribute\texttt{>}, which are parsed and stored in the internal file 
structure in ADIOS. The element \texttt{<}gwrite\texttt{>} only affects the execution 
of python scripts (see Chapter \ref{chapter-gpp}). 
Any content (usually comments, conditional statements, 
or loop statements) in the value of attribute ``src'' is copied identically into 
generated pre-processing files. This is the way to write a subset of variables optionally
depending on a logical expression in the source-code. Declaration
\begin{lstlisting}[language=XML]
<gwrite src="if (have_ions==1) then"/>
    ...
<gwrite src="endif"/>
\end{lstlisting}

Required:
\begin{itemize}
\item src -  any statement that needs to be added into the source code. This code will 
be inserted into the source code, and must be able to be compiled in the host language, 
C or Fortran. 
\end{itemize}

\subsection{Global arrays}
\label{section-xml-global-arrays}
The \textbf{global-bounds} element is an optional nested element for the adios-group. 
It specifies the global space and offsets within that space for the enclosed variable 
elements. In the case of writing to a shared file, the global-bounds information 
is recorded in BP file and can be interpreted by converters or other postprocessing 
tools or used to write out either HDF5 or NetCDF files by using PHDF5 or the PnetCDF 
method.

\subsubsection{Declaration}
\begin{lstlisting}[language=XML]
<global-bounds dimensions="nx_g, ny_g" offsets="nx_o,0"/>
    ... variable declarations ... 
</global-bounds>
\end{lstlisting}

Required:
\begin{itemize}
\item dimensions - the dimension of global space
\item offsets - the offset of the data set in global space
\end{itemize}

Any variables used in the global-bounds element for dimensions or offsets declaration 
need to be defined in the same adios-group as either variables or attributes. 

For detailed global arrays use, see the example illustrated in 
Section \ref{section-cprogramming-globalarrays}.


\subsection{Time-index}

ADIOS allows a dataset to be expanded in the space domain given by global bounds 
and in time domain. It is very common for scientific applications to write out 
a monitoring file at regular intervals. The file usually contains a group of time-based 
variables that have undetermined dimensional value on the time axis. ADIOS is Similar 
to NetCDF in that it accumulates the time-index in terms of the number of records, 
which theoretically can be added to infinity.

If any of variables in an adios group are time based, they can be marked out by 
adding the time-index variable as another dimension value. Note, that with the new
read API, time is not represented as an extra dimension at reading. If you write a
2D variable over multiple steps, you will see it still as a 2D variable, which has
multiple steps. Nevertheless, multiple, consecutive steps can be read at once into
a contigous memory with one read request, if needed.


\section{Transport method} 

The method element provides the 
hook between the adios-group and the transport methods. The user employs a different 
transport method simply by changing the method attribute of the method element. 
If more than one method element is provided for a given group, each element will 
be invoked in the order specified. This neatly gives triggering opportunities for 
workflows. To trigger a workflow once the analysis data set has been written to 
disk, the user makes two element entries for the analysis adios-group. The first 
indicates how to write to disk, and the second performs the trigger for the workflow 
system. No recompilation, relinking, or any other code changes are required for 
any of these changes to the XML file.

\subsection{Declaration}

The transport element is used to specify the mapping of an I/O transport method, 
including optional initialization parameters, to the respective adios-group. 
Either the term \verb+transport+ or \verb+method+ can be used for this element. 
There are two major attributes required for the method element: 
\begin{lstlisting}[language=XML]
<transport group="restart"
    method="MPI"
    priority="1" 
    base-path="/proj/proj034/simdata/run256" 
    iteration="100"/>
\end{lstlisting}

Required:
\begin{itemize}
\item group - corresponds to an adios-group specified earlier in the file.
\item method - a string indicating a transport method to use with the associated adios-group
\end{itemize}

Optional: 
\begin{itemize}
\item priority - a numeric priority for the I/O method to better schedule this write with 
others that may be pending currently
\item base-path - the root directory to use when writing to disk or similar purposes
\item iterations - a number of iterations between writes of this group used to gauge how 
quickly this data should be evacuated from the compute node
\end{itemize}

\subsection{Methods list}
As the componentization of the IO substrate, ADIOS supports a list of transport 
methods, described in Chapter \ref{chapter-methods}:

\begin{itemize}
\item NULL
\item POSIX
\item MPI
\item MPI\_LUSTRE
\item MPI\_AGGR (or MPI\_AMR)
\item VAR\_MERGE
\item PHDF5 (for Parallel HDF5)
\item NC4PAR (for Parallel NetCDF4)
\item DATASPACES (or DART)
\item DIMES
\item FLEXPATH
\end{itemize}



\section{Buffer specification}
\label{section-xml-buffers-pecification}
The buffer element defines the attributes for internal buffer size and creating 
time that apply to the whole application (Listing~\ref{list-example-xml}). The attribute allocate-time 
is identified as being either{\small  ``}now'' or ``oncall'' to indicate when the 
buffer should be allocated. An ``oncall'' attribute waits until the programmer 
decides that all memory needed for calculation has been allocated. It then calls 
upon ADIOS to allocate buffer. There are two alternative attributes for users to 
define the buffer size: MB and free-memory-percentage. 

\subsection{Declaration}
\begin{lstlisting}[language=XML]
<buffer size-MB="100"
    allocate-time="now" />
\end{lstlisting}

Required:
\begin{itemize}
\item size-MB - the user-defined size of  buffer in megabytes. ADIOS can at most allocate 
from compute nodes. It is exclusive with free-memory-percentage.
\item free-memory percentage - the user-defined percentage from 0 to 100\% of free memory 
available on the machine. It is exclusive with size-MB.
\item allocate-time - indicates when the buffer should be allocated
\end{itemize}

\section{Enabling Histogram}

ADIOS 1.2 has the ability to {\color{color01} compute a histogram of the given 
variable's data values at write time}. This is specified via the \textbf{\texttt{<}analysis\texttt{>}} 
tag in the XML file. The parameters \texttt{"}\textbf{adios-group}\texttt{"} and 
\texttt{"}\textbf{var}\texttt{"} specify the variable for which the histogram is 
to be performed. \texttt{"}\textbf{var}\texttt{"} is the name of the variable and 
\texttt{"}\textbf{adios-group}\texttt{"} is the name of the adios group to which 
the variable belongs to.

\subsection{Declaration}
The histogram binning intervals can be input in two ways via the XML file:
\begin{itemize}
\item By listing the break points as a list of comma separated values 
in the parameter \texttt{"}break-points\texttt{"} 
\begin{lstlisting}[language=XML]
<analysis adios-group="temperature" var="temperature"
    break-points="0, 100, 200, 300" />
\end{lstlisting}

\item By specifying the boundaries of the breaks, and the number 
of intervals between variable's min and max values
\begin{lstlisting}[language=XML]
<analysis adios-group="temperature" var="temperature"
    min="0" max="300" count="3"/>
\end{lstlisting}
\end{itemize}

Both inputs create the bins (-Inf, 0), [0, 100), [100, 200), [200, 300), [300, 
Inf). For this example, the final set of frequencies for these 5 binning intervals 
will be calculated.

Required:
\begin{itemize}
\item adios-group - corresponds to an adios-group specified earlier in the file.
\item var - corresponds to a variable in adios-group specified earlier in the file.
\end{itemize}

Optional:
\begin{itemize}
\item break-points - list of comma separated values \textbf{sorted} in ascending order
\item min  - minimum value of the binning boundary
\item max - maximum value of the binning boundary 
(it should be greater than min)
\item count - number of break points between the min and max values 
\end{itemize}

A valid set of binning intervals must be provided either by specifying \texttt{"}min,\texttt{"} 
\texttt{"}max,\texttt{"} and \texttt{"}count\texttt{"} parameters or by providing 
the \texttt{"}break-points.\texttt{"} The intervals given under \texttt{"}break-points\texttt{"} 
will take precedence when calculating the histogram intervals, if \texttt{"}min,\texttt{"} 
\texttt{"}max,\texttt{"} and \texttt{"}count\texttt{"} as well as ``break-points'' 
are provided.

\section{An Example XML file}

\begin{lstlisting}[language=XML, caption={Example XML file.}, label=list-example-xml]
<adios-config host-language="C">
    <adios-group name="temperature" coordination-communicator="comm">
        <var name="NX" type="integer"/>
        <var name="t" type="double" dimensions="NX"/>
        <attribute name="recorded date" path="/" value="Sep 19, 2008" 
            type="string"/> 
        
        <!-- conditional writing of a variable -->
        <gwrite src="if (want_x) {" />
            <var name="x" type="integer" dimensions="NX"/>
        <gwrite src="}" />
    </adios-group>

    
    <method group=" temperature " method="MPI"/>
    
    <buffer size-MB="1" allocate-time="now"/>
    <analysis adios-group="temperature" var="t" break-points="0, 100, 200, 300"/> 
</adios-config>
\end{lstlisting}

