\chapter{Installation}

\section{Obtaining ADIOS}

You can download the latest version from the following website 

\begin{lstlisting}[language={}]
http://www.olcf.ornl.gov/center-projects/adios
\end{lstlisting}


\section{Quick Installation}

To get started with ADIOS, the following steps can be used to configure, build, 
test, and install the ADIOS library, header files, and support programs. 

\begin{lstlisting}
cd trunk/
./configure -prefix=<install-dir> --with-mxml=<mxml-location>
make
make install
\end{lstlisting}

Note: There is a runconf batch script in the trunk set up for our machines. Studying 
it can help you setting up the appropriate environment variables and configure 
options for your system.

\subsection{Linux cluster}

The following is a snapshot of the batch scripts on Sith, an Intel-based Infiniband 
cluster running Linux:

\begin{lstlisting}
export MPICC=mpicc
export MPICXX=mpiCC
export MPIFC=mpif90
export CC=pgcc
export CXX=pgCC
export FC=pgf90
export CFLAGS="-fPIC"

./configure --prefix = <location for ADIOS software installation>
            --with-mxml=<location of mini-xml installation>
            --with-hdf5=<location of HDF5 installation>
            --with-netcdf=<location of netCDF installation>
\end{lstlisting}


The compiler pointed by MPICC is used to build all the parallel codes and tools 
using MPI, while the compiler pointed by CC is used to build the sequential tools. 
In practice, mpicc uses the compiler pointed by CC and adds the MPI library automatically. 
On clusters, this makes no real difference, but on Bluegene, or Cray XT, parallel 
codes are built for compute nodes, while the sequential tools are built for the 
login nodes. The -fPIC compiler flag is needed only if you build the 
Matlab language bindings later.


\subsection{Cray XT5}

To install ADIOS on a Cray XT5, the right compiler commands and configure flags 
need to be set. The required commands for ADIOS installation on Jaguar are as follows:

\begin{lstlisting}
export CC=cc
export CXX=CC
export FC=ftn
./configure --prefix = <location for ADIOS software installation>
            --with-mxml=<location of mini-xml installation>
            --with-hdf5=<location of HDF5 installation>
            --with-netcdf=<location of netCDF installation>
\end{lstlisting}


\section{ADIOS Dependencies}

\subsection{Mini-XML parser (required)}

The Mini-XML library is used to parse XML configuration files. Mini-XML can be 
downloaded from \url{http://www.minixml.org/software.php}


\subsection{MPI and MPI-IO (required)}

MPI and MPI-IO is required for ADIOS.

Currently, most large-scale scientific applications rely on the Message Passing 
Interface (MPI) library to implement communication among processes. For instance, 
when the Portable Operating System Interface (POSIX) is used as transport method, 
the rank of each processor in the same communication group, which needs to be retrieved 
by the certain MPI APIs, is commonly used in defining the output files. MPI-IO 
can also be considered the most generic I/O library on large-scale platforms. 

\subsection{Python (required)}

The XML processing utility \verb+utils/gpp/gpp.py+ is a code written in python using xml.dom.minidom. 
It is used to generate C or Fortran code from the XML configuration files that 
can be included in the application source code.  Examples and tests will not build 
without Python. 

\subsection{Fortran90 compiler (optional)}

The Fortran~90 interface and example codes are compiled only if there is an f90 
compiler available. By default it is required but you can disable it with the option 
\verb+--disable-fortran+.

\subsection{Serial NetCDF-3 (optional)}

The bp2ncd converter utility to NetCDF format is built only if NetCDF~is available. 
 Currently ADIOS uses the NetCDF-3 library. Use the option \verb+--with-netcdf=<path>+ 
or ensure that the \verb+NETCDF_DIR+ environment variable is set before configuring ADIOS.

\subsection{Serial HDF5 (optional)}

The bp2h5 converter utility to HDF5 format is built only if a HDF5 library is available. 
Currently ADIOS uses the 1.6 version of the HDF5 API but it can be built and used 
with the 1.8.x version of the HDF5 library too. Use the option \verb+--with-hdf5=<path>+ 
when configuring ADIOS.

\subsection{Lustreapi (optional)}

The Lustreapi library is used internally by \verb+MPI_LUSTRE+ and \verb+MPI_AMR+ method to 
figure out Lustre parameters such as stripe count and stripe size.  Without giving 
this option, users are expected to manually set Lustre parameters from ADIOS XML 
configuration file (see \verb+MPI_LUSTRE+ and \verb+MPI_AMR+ method). 
Use the configuration option
\verb+--with-lustre=<path>+ to define the path to this library.

\subsection{Staging transport methods (optional)}

In ADIOS 1.4, a transport method using the DataSpaces library (Rutgers University) 
is available for memory-to-memory transfer (staging) of data between two 
applications. 

\subsubsection{Networking libraries for staging}

Staging methods use Remote Direct Memory Access (RDMA) operations, supported by specific libraries 
on various systems. 

\vspace*{6pt}
\noindent {\bf Infiniband. } 
If you have an Infininband network with \verb+ibverbs+ and \verb+rdmacm+ libraries installed, you can configure ADIOS to use it for staging methods with the option
\verb+--with-infiniband=DIR+  to define the path to the Infiniband libraries. 

\vspace*{6pt}
\noindent {\bf Cray Gemini network. }
On newer Cray machines (XK6 and XE6) with the Gemini network, the \verb+PMI+ and \verb+uGNI+ libraries are used by the staging methods. Configure ADIOS with the options

\begin{lstlisting}
--with-cray-pmi=/opt/cray/pmi/default \
--with-cray-ugni-incdir=/opt/cray/gni-headers/default/include \
--with-cray-ugni-libdir=/opt/cray/ugni/default/lib
\end{lstlisting}

\vspace*{6pt}
\noindent {\bf Portals. }
Portals is an RDMA library from Sandia Labs, and it has been used on Cray XT5 machines with Seastar networks. Configure ADIOS with the option

\verb+--with-portals=DIR      Location of Portals (yes/no/path_to_portals)+

\subsubsection{DataSpaces staging methods}
The DataSpaces model provides a separate server running on separate compute nodes, into/from which data can be written/read with a geometrical (3D) abstraction. It is an efficient way to stage data from one application to another in an asynchronous (and very fast) way. Multiple steps of data outputs can be stored, limited only by the available memory. DataSpaces can be downloaded from \url{http://www.dataspaces.org}

\noindent Build the DataSpaces method with the option:

\begin{lstlisting}
--with-dataspaces=DIR  Build the DATASPACES transport method. Point to the
                      DATASPACES installation.
--with-dataspaces-incdir=<location of dataspaces includes>
--with-dataspaces-libdir=<location of dataspaces library>
\end{lstlisting}

 
\subsection{Read-only installation}

If you just want the read API to be compiled for reading BP files, use the \verb+--disable-write+ option.

\subsection{PHDF5 (optional)}

The transport method writing files in the Parallel HDF5 format is built only if 
a parallel version of the HDF5 library is available. You need to use the 
option \verb+--with-phdf5=<path>+ to build this transport method. 

Note: Do not expect better performance with ADIOS/PHDF5 than with PHDF5 itself. ADIOS does not write differently to a HDF5 formatted file, it simply uses PHDF5 function calls to write out data. Also good to know, that the method in ADIOS uses the collective function calls, that requires that every process participates in the writing of each variable. 

If you define Parallel HDF5 and do not define serial HDF5, then bp2h5 will be built 
with the parallel library. 
Note that if you build this transport method, ADIOS will depend on PHDF5 when you 
link any application with ADIOS even if your application does not intend to 
use this method. 
If you have problems compiling ADIOS with PHDF5 due to missing flags or libraries, 
you can define them using 

\begin{lstlisting}
--with-phdf5-incdir=<path>,
--with-phdf5-libdir=<path> and 
--with-phdf5-libs=<link time flags and libraries>
\end{lstlisting}

\subsection{NetCDF-4 Parallel (optional)}

The NC4 transport method writes files using the NetCDF-4 library which in turn 
is based on the parallel HDF5 library. You need to use the option 
\verb+--with-nc4par=<path>+ to build this transport method. 
You also need to provide the parallel HDF5 library. 

Note: Do not expect better performance with ADIOS/NC4 than with NC4 itself. ADIOS does not write differently to a HDF5 formatted file, it simply uses NC4 function calls to write out data. Also good to know, that this method requires that every process participates in the writing of each variable. 

\section{Full Installation}

The following list is the complete set of options that can be used with 
configure to build ADIOS and its support utilities:

\begin{lstlisting}
--help              print the usage of ./configure command}
--with-tags[=TAGS]  include additional configurations [automatic]
--with-mxml=DIR     Location of Mini-XML library
--with-infiniband=DIR      Location of Infiniband
--with-portals=DIR      Location of Portals (yes/no/path_to_portals)
--with-cray-pmi=<location of CRAY_PMI installation>
--with-cray-pmi-incdir=<location of CRAY_PMI includes>
--with-cray-pmi-libdir=<location of CRAY_PMI library>
--with-cray-pmi-libs=<linker flags besides -L<cray-pmi-libdir>, e.g. -lpmi
--with-cray-ugni=<location of CRAY UGNI installation>
--with-cray-ugni-incdir=<location of CRAY UGNI includes>
--with-cray-ugni-libdir=<location of CRAY UGNI library>
--with-cray-ugni-libs=<linker flags besides -L<cray-ugni-libdir>, e.g. -lugni
--with-hdf5=<location of HDF5 installation>
--with-hdf5-incdir=<location of HDF5 includes>
--with-hdf5-libdir=<location of HDF5 library>
--with-phdf5=<location of PHDF5 installation>
--with-phdf5-incdir=<location of PHDF5 includes>
--with-phdf5-libdir=<location of PHDF5 library>
--with-netcdf=<location of NetCDF installation>
--with-netcdf-incdir=<location of NetCDF includes>
--with-netcdf-libdir=<location of NetCDF library>
--with-nc4par=<location of NetCDF 4 Parallel installation>
--with-nc4par-incdir=<location of NetCDF 4 Parallel includes>
--with-nc4par-libdir=<location of NetCDF 4 Parallel library>
--with-nc4par-libs=<linker flags besides -L<nc4par_libdir>, e.g. -lnetcdf
--with-dataspaces=<location of DataSpaces installation>
--with-dataspaces-incdir=<location of DataSpaces includes>
--with-dataspaces-libdir=<location of DataSpaces library>
--with-lustre=<location of Lustreapi library>
\end{lstlisting}

Some influential environment variables are lists below:

\begin{lstlisting}
CC        C compiler command
CFLAGS    C compiler flags
LDFLAGS   linker flags, e.g. -L<lib dir> if you have libraries 
          in a nonstandard directory <lib dir>
CPPFLAGS  C/C++ preprocessor flags, e.g. -I<include dir> if you
          have headers in a nonstandard directory <include dir>
CPP       C preprocessor
CXX       C++ compiler command
CXXFLAGS  C++ compiler flags
FC        Fortran compiler command
FCFLAGS   Fortran compiler flags
CXXCPP    C++ preprocessor
F77       Fortran 77 compiler command
FFLAGS    Fortran 77 compiler flags
MPICC     MPI C compiler command
MPIFC     MPI Fortran compiler command
\end{lstlisting}


\section{Compiling applications using ADIOS}
\label{section:installation_compiling_apps}

ADIOS configuration creates a text file that contains the flags and library dependencies 
that should be used when compiling/linking user applications that use ADIOS. This 
file is installed as \verb+bin/adios_config.flags+ under the installation directory by 
make install. A script, named \verb+adios_config+ is also installed that can print out 
selected flags. In a Makefile, if you set \verb+ADIOS_DIR+ to the installation directory 
of ADIOS, you can set the flags for building your code flexibly as shown below 
for a Fortran application: 

\begin{lstlisting}
override ADIOS_DIR := <your ADIOS installation directory>
override ADIOS_INC := $(shell ${ADIOS_DIR}/bin/adios_config -c -f)
override ADIOS_FLIB := $(shell ${ADIOS_DIR}/bin/adios_config -l -f)

example.o : example.F90
        ${FC} -g -c ${ADIOS_INC} example.F90 $<

example: example.o
        ${FC} -g -o example example.o ${ADIOS_FLIB} 
\end{lstlisting}

The example above is for using write (and read) in a Fortran + MPI application. However, several libraries are built for specific uses:

\begin{itemize}
\item \verb+libadios.a            +   MPI + C/C++ using ADIOS to write and optionally read data
\item \verb+libadiosf.a           +   MPI + Fortran using ADIOS to write and optionally read data
\item \verb+libadios_nompi.a      +   C/C++ without MPI
\item \verb+libadiosread.a        +   MPI + C/C++ using ADIOS to only read data
\item \verb+libadiosreadf.a       +   MPI + Fortran using ADIOS to only read data
\item \verb+libadiosread_nompi.a  +   C/C++ without MPI, using ADIOS to only read data
\item \verb+libadiosreadf_nompi.a +   Fortran without MPI, using ADIOS to only read data
\item \verb+libadiosf_v1.a        +   MPI + Fortran using ADIOS to write and, with the old read API to read data
\item \verb+libadiosreadf_v1.a    +   MPI + Fortran using ADIOS old read API to read data
\end{itemize}

\subsection{Sequential applications}

Use the \verb+-D_NOMPI+ pre-processor flag to compile your application 
for a sequential build. ADIOS has a dummy MPI library, \verb+mpidummy.h+, that re-defines 
all MPI constructs necessary to run ADIOS without MPI. You can declare

\verb+MPI_Comm comm;+

in your sequential code to pass it on to functions that require an \verb+MPI_Comm+ variable.

If you want to write a C/C++ parallel code using MPI, but also want to provide it as a
sequential tool on a login-node without modifying the source code, then write your 
application as MPI, do not include \verb+mpi.h+ but include 
\verb+adios.h+ or \verb+adios_read.h+.
for the sequential build.
\verb+adios.h+/\verb+adios_read.h+ include the appropriate header file 
\verb+mpi.h+ or \verb+mpidummy.h+ 
(the latter provided by ADIOS) depending on which version you want to build. 


\section{Language bindings}

ADIOS comes with various bindings to languages, that are not built with the automake tools discussed above. After building ADIOS, these bindings have to be manually built.

\subsection{Support for Matlab}
\label{section-install-matlab}

Matlab requires ADIOS be built with the GNU C compiler. It also requires relocatable 
codes, so you need to add the -fPIC flag to CFLAGS before configuring ADIOS. 
You need to compile it with Matlab's MEX compiler after the make and copy the files 
manually to somewhere where Matlab can see them or set the \verb+MATLABPATH+ to this 
directory to let Matlab know where to look for the bindings.

\begin{lstlisting}
cd wrappers/matlab
make matlab
\end{lstlisting}


\subsection{Support for Java}
\label{section-install-java}

ADIOS provides a Java language binding implemented by the Java Native Interface (JNI).
The program can be built with CMake (\url{http://www.cmake.org/}) which will detect your ADIOS installation and related programs and libraries. With CMake, you can create a build directory and run cmake pointing the Java wrapper source directory (wrappers/java) containing CMakeLists.txt. For example, 
\begin{lstlisting}
cd wrappers/java
mkdir build
cd build
cmake ..
\end{lstlisting}

CMake will search installed ADIOS libraries, Java, JNI, MPI libraries (if
needed), etc. Once completed, type \verb+make+ to build. If you need verbose output, you type as follows:
\begin{lstlisting}
make VERBOSE=1
\end{lstlisting}

After successful building, you will see libAdiosJava.so (or
libAdiosJava.dylib in Mac) and AdiosJava.jar. Those two files will be needed to use in Java. Detailed instructions for using this Java binding will be discussed in Section~\ref{section-bindings-java}.

If you want to install those files, type the following:
\begin{lstlisting}
make install
\end{lstlisting}

The default installation directory is \verb+/usr/local+. You can change by
specifying \verb+CMAKE_INSTALL_PREFIX+ value;
\begin{lstlisting}
cmake -DCMAKE_INSTALL_PREFIX=/path/to/install /dir/to/source
\end{lstlisting}

Or, you can use the \verb+ccmake+ command, the CMake curses interface. Please refer to the CMake documents for more detailed instructions.

This program contains a few test programs. To run testing after building,
type the following command:
\begin{lstlisting}
make test
\end{lstlisting}

If you need a verbose output, type the following
\begin{lstlisting}
ctest -V
\end{lstlisting}

\subsection{Support for Numpy}
\label{section-install-numpy}

ADIOS also provides a Python/Numpy language binding. The source code is located in \verb+wrappers/numpy+. This module is developed by Cython.

Like the Java binding, this Python/Numpy wrapper can be built by using CMake (\url{http://www.cmake.org/}). You can create a build directory and run cmake by pointing the source directory. For example, 
\begin{lstlisting}
cd wrappers/numpy
mkdir build
cd build
cmake ..
\end{lstlisting}

CMake will search installed ADIOS library and Python/Numpy. Once completed, type \verb+make+ to build or type the following for the verbose output.
\begin{lstlisting}
make VERBOSE=1
\end{lstlisting}

After successful building, you will see adios.so. This file can be
loaded in Python. Detailed instructions for using this module in Python will be discussed in Section~\ref{section-bindings-numpy}.

This program contains a few test programs. To run testing after building,
type the following command:
\begin{lstlisting}
make test
\end{lstlisting}

If you need a verbose output, type the following
\begin{lstlisting}
ctest -V
\end{lstlisting}


