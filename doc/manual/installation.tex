\chapter{Installation}

\section{Obtaining ADIOS}

You can download the latest version from the following website 

\begin{lstlisting}[language={}]
http://www.olcf.ornl.gov/center-projects/adios
\end{lstlisting}


\section{Quick Installation}

To get started with ADIOS, the following steps can be used to configure, build, 
test, and install the ADIOS library, header files, and support programs. 

\begin{lstlisting}
cd trunk/
./configure -prefix=<install-dir> --with-mxml=<mxml-location>
make
make install
\end{lstlisting}

Note: There is a runconf batch script in the trunk set up for our machines. Studying 
it can help you setting up the appropriate environment variables and configure 
options for your system.

\subsection{Linux cluster}

The following is a snapshot of the batch scripts on Ewok, an Intel-based Infiniband 
cluster running Linux:

\begin{lstlisting}
export MPICC=mpicc
export MPICXX=mpiCC
export MPIFC=mpif90
export CC=pgcc
export CXX=pgCC
export FC=pgf90
export CFLAGS="-fPIC"

./configure --prefix = <location for ADIOS software installation>
            --with-mxml=<location of mini-xml installation>
            --with-hdf5=<location of HDF5 installation>
            --with-netcdf=<location of netCDF installation>
\end{lstlisting}


The compiler pointed by MPICC is used to build all the parallel codes and tools 
using MPI, while the compiler pointed by CC is used to build the sequential tools. 
In practice, mpicc uses the compiler pointed by CC and adds the MPI library automatically. 
On clusters, this makes no real difference, but on Bluegene, or Cray XT, parallel 
codes are built for compute nodes, while the sequential tools are built for the 
login nodes. The -fPIC compiler flag is needed only if you build the Matlab tools.


\subsection{Cray XT5}

To install ADIOS on a Cray XT5, the right compiler commands and configure flags 
need to be set. The required commands for ADIOS installation on Jaguar are as follows:

\begin{lstlisting}
export CC=cc
export CXX=CC
export FC=ftn
./configure --prefix = <location for ADIOS software installation>
            --with-mxml=<location of mini-xml installation>
            --with-hdf5=<location of HDF5 installation>
            --with-netcdf=<location of netCDF installation>
\end{lstlisting}


\subsection{Support for Matlab}

Matlab requires ADIOS be built with the GNU C compiler. It also requires relocatable 
codes, so you need to add the -fPIC flag to CFLAGS before configuring ADIOS. The 
matlab reader is not built automatically at make and is not installed with ADIOS. 
You need to compile it with Matlab's MEX compiler after the make and copy the files 
manually to somewhere where Matlab can see them.

\begin{lstlisting}
cd tools/matlab
make matlab
\end{lstlisting}


\section{ADIOS Dependencies}

\subsection{Mini-XML parser (required)}

The Mini-XML library is used to parse XML configuration files. Mini-XML can be 
downloaded from 

\begin{lstlisting}[language={}]
http://www.minixml.org/software.php
\end{lstlisting}


\subsection{MPI and MPI-IO (required)}

MPI and MPI-IO is required for ADIOS.

Currently, most large-scale scientific applications rely on the Message Passing 
Interface (MPI) library to implement communication among processes. For instance, 
when the Portable Operating System Interface (POSIX) is used as transport method, 
the rank of each processor in the same communication group, which needs to be retrieved 
by the certain MPI APIs, is commonly used in defining the output files. MPI-IO 
can also be considered the most generic I/O library on large-scale platforms. 

\subsection{Python (required)}

The XML processing utility \verb+utils/gpp/gpp.py+ is a code written in python using xml.dom.minidom. 
It is used to generate C or Fortran code from the XML configuration files that 
can be included in the application source code.  Examples and tests will not build 
without Python. 

\subsection{Fortran90 compiler (optional)}

The Fortran~90 interface and example codes are compiled only if there is an f90 
compiler available. By default it is required but you can disable it with the option 
\verb+--disable-fortran+.

\subsection{Serial NetCDF-3 (optional)}

The bp2ncd converter utility to NetCDF format is built only if NetCDF~is available. 
 Currently ADIOS uses the NetCDF-3 library. Use the option \verb+--with-netcdf=<path>+ 
or ensure that the \verb+NETCDF_DIR+ environment variable is set before configuring ADIOS.

\subsection{Serial HDF5 (optional)}

The bp2h5 converter utility to HDF5 format is built only if a HDF5 library is available. 
Currently ADIOS uses the 1.6 version of the HDF5 API but it can be built and used 
with the 1.8.x version of the HDF5 library too. Use the option \verb+--with-hdf5=<path>+ 
when configuring ADIOS.

\subsection{PHDF5 (optional)}

The transport method writing files in the Parallel HDF5 format is built only if 
a parallel version of the HDF5 library is (also) available. You need to use the 
option \verb+--withh-phdf5=<path>+ to build this transport method. 

If you define Parallel HDF5 and do not define serial HDF5, then bp2h5 will be built 
with the parallel library. 

Note that if you build this transport method, ADIOS will depend on PHDF5 when you 
link any application with ADIOS even if you the application does not intend to 
use this method. 

If you have problems compiling ADIOS with PHDF5 due to missing flags or libraries, 
you can define them using 

\begin{lstlisting}
--with-phdf5-incdir=<path>,
--with-phdf5-libdir=<path> and 
--with-phdf5-libs=<link time flags and libraries>
\end{lstlisting}

\subsection{NetCDF-4 Parallel (optional)}

The NC4 transport method writes files using the NetCDF-4 library which in turn 
is based on the parallel HDF5 library. You need to use the option 
\verb+--with-nc4par=<path>+ to build this transport method. Also, 
you need the parallel HDF5 library. 

\subsection{Lustreapi (optional)}

The Lustreapi library is used internally by \verb+MPI_LUSTRE+ and \verb+MPI_AMR+ method to 
figure out Lustre parameters such stripe count and stripe size.  Without giving 
this option, users are expected to manually set Lustre parameters from ADIOS XML 
configuration file (see \verb+MPI_LUSTRE+ and \verb+MPI_AMR+ method). Use the configuration option
\verb+--with-lustre=<path>+ to define the path to this library.

\subsection{Read-only installation}

If you just want the read API to be compiled for reading BP files, use the \verb+--disable-write+ option.

\section{Full Installation}

The following list is the complete set of options that can be used with 
configure to build ADIOS and its support utilities:

\begin{lstlisting}
--help              print the usage of ./configure command}
--with-tags[=TAGS]  include additional configurations [automatic]
--with-mxml=DIR     Location of Mini-XML library
--with-hdf5=<location of HDF5 installation>
--with-hdf5-incdir=<location of HDF5 includes>
--with-hdf5-libdir=<location of HDF5 library>
--with-phdf5=<location of PHDF5 installation>
--with-phdf5-incdir=<location of PHDF5 includes>
--with-phdf5-libdir=<location of PHDF5 library>
--with-netcdf=<location of NetCDF installation>
--with-netcdf-incdir=<location of NetCDF includes>
--with-netcdf-libdir=<location of NetCDF library>
--with-nc4par=<location of NetCDF 4 Parallel installation>
--with-nc4par-incdir=<location of NetCDF 4 Parallel includes>
--with-nc4par-libdir=<location of NetCDF 4 Parallel library>
--with-nc4par-libs=<linker flags besides -L<nc4par_libdir>, e.g. -lnetcdf
--with-lustre=<location of Lustreapi library>
\end{lstlisting}

Some influential environment variables are lists below:

\begin{lstlisting}
CC        C compiler command
CFLAGS    C compiler flags
LDFLAGS   linker flags, e.g. -L<lib dir> if you have libraries 
          in a nonstandard directory <lib dir>
CPPFLAGS  C/C++ preprocessor flags, e.g. -I<include dir> if you
          have headers in a nonstandard directory <include dir>
CPP       C preprocessor
CXX       C++ compiler command
CXXFLAGS  C++ compiler flags
FC        Fortran compiler command
FCFLAGS   Fortran compiler flags
CXXCPP    C++ preprocessor
F77       Fortran 77 compiler command
FFLAGS    Fortran 77 compiler flags
MPICC     MPI C compiler command
MPIFC     MPI Fortran compiler command
\end{lstlisting}


\section{Compiling applications using ADIOS}
\label{section:installation_compiling_apps}

Adios configuration creates a text file that contains the flags and library dependencies 
that should be used when compiling/linking user applications that use ADIOS. This 
file is installed as \verb+bin/adios_config.flags+ under the installation directory by 
make install. A script, named \verb+adios_config+ is also installed that can print out 
selected flags. In a Makefile, if you set \verb+ADIOS_DIR+ to the installation directory 
of ADIOS, you can set the flags for building your code flexibly as shown below 
for a Fortran application: 

\begin{lstlisting}
override ADIOS_DIR := <your ADIOS installation directory>
override ADIOS_INC := $(shell ${ADIOS_DIR}/bin/adios_config -c -f)
override ADIOS_FLIB := $(shell ${ADIOS_DIR}/bin/adios_config -l -f)

example.o : example.F90
        ${FC} -g -c ${ADIOS_INC} example.F90 $<

example: example.o
        ${FC} -g -o example example.o ${ADIOS_FLIB} 
\end{lstlisting}

