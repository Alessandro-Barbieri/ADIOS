
\vspace{24pt}
\begin{flushright}
\textbf{ORNL/TM-2009/100\label{OLEHLINK6}}
\end{flushright}

\vspace{60pt}
{\huge \textbf{\adiosversion User's Manual}}

\vspace{36pt}
\textbf{November 2017\pagebreak{}}


\begin{longtable}{|p{4.443in}|p{0.057in}|}
\hline

\begin{center}
{\small \textbf{DOCUMENT AVAILABILITY}}
\end{center}


{\small Reports produced after January 1, 1996, are generally available free via 
the U.S. Department of Energy (DOE) Information Bridge:}


\leftskip=18pt
{\small \textbf{Web site:}}{\small  http://www.osti.gov/bridge}


\leftskip=0pt
{\small Reports produced before January 1, 1996, may be purchased by members of 
the public from the following source:}


\parindent=18pt
{\small National Technical Information Service}

{\small 5285 Port Royal Road}

{\small Springfield, VA 22161}

{\small \textit{\textbf{Telephone:}}}{\small  703-605-6000 (1-800-553-6847)}

{\small \textit{\textbf{TDD:}}}{\small  703-487-4639}

{\small \textit{\textbf{Fax:}}}{\small  703-605-6900}

{\small \textit{\textbf{E-mail:}}}{\small  info@ntis.fedworld.gov}

{\small \textit{\textbf{Web site:}}}{\small  http://www.ntis.gov/support/ordernowabout.htm}


\parindent=0pt
{\small Reports are available to DOE employees, DOE contractors, Energy Technology 
Data Exchange (ETDE) representatives, and International Nuclear Information System 
(INIS) representatives from the following source:}


\parindent=18pt
{\small Office of Scientific and Technical Information}

{\small P.O. Box 62}

{\small Oak Ridge, TN 37831}

{\small \textit{\textbf{Telephone:}}}{\small  865-576-8401}

{\small \textit{\textbf{Fax:}}}{\small  865-576-5728}

{\small \textit{\textbf{E-mail:}}}{\small  reports@adonis.osti.gov}

\leftskip=18pt
\parindent=0pt
{\small \textit{\textbf{Web site:}}}{\small  http://www.osti.gov/contact.html}

\\\hline
\end{longtable}

%\vspace{48pt}
\begin{longtable}{|p{4.443in}|p{0.057in}|}
\hline
% ROW 1
\begin{minipage}[t]{4.443in}\raggedright %\linebreak
{\small This report was prepared as an account of work sponsored by an agency of 
the United States Government. Neither the United States government nor any agency 
thereof, nor any of their employees, makes any warranty, express or implied, or 
assumes any legal liability or responsibility for the accuracy, completeness, or 
usefulness of any information, apparatus, product, or process disclosed, or represents 
that its use would not infringe privately owned rights. Reference herein to any 
specific commercial product, process, or service by trade name, trademark, manufacturer, 
or otherwise, does not necessarily constitute or imply its endorsement, recommendation, 
or favoring by the United States Government or any agency thereof. The views and 
opinions of authors expressed herein do not necessarily state or reflect those 
of the United States Government or any agency thereof.}\end{minipage}\\
\hline
\end{longtable}
\pagebreak{}

\vspace{12pt}
\begin{flushright}
\textbf{ORNL/TM-2009/100}
\end{flushright}

%\vspace{36pt}
\begin{center}
{\Large \textbf{\adiosversion USER'S MANUAL}}

\vspace{60pt}
Prepared for the

%\vspace{12pt}
Office of Science

%\vspace{12pt}
U.S. Department of Energy

\vspace{60pt}
Authors

\vspace{6pt}
Norbert Podhorszki, Qing Liu, Jeremy Logan, Jingqing Mu, Hasan Abbasi, Jong-Youl Choi, Scott A. Klasky

\vspace{30pt}
Contributors 

\vspace{6pt}
Jay Lofstead, Steve Hodson, Fang Zheng, Matthew Wolf, Todd Kordenbrock, Nagiza Samatova, Jay Dayal, Yuan Tian, David Boyuka

\vspace{72pt}
November  2016

\vspace{72pt}
Prepared by

%\vspace{24pt}
OAK RIDGE NATIONAL LABORATORY

%\vspace{12pt}
Oak Ridge, Tennessee 37831-6070

%\vspace{12pt}
managed by

%\vspace{12pt}
UT-BATTELLE, LLC

%\vspace{12pt}
for the

%\vspace{12pt}
U.S. DEPARTMENT OF ENERGY

%\vspace{12pt}
under contract DE-AC05-00OR22725

%\vspace{42pt}
\end{center}


\newpage

\tableofcontents


\newpage

\listoffigures


\newpage

\vspace{66pt}
\textbf{Abbreviations}

\begin{description}
\item[ADIOS]  Adaptive Input/Output System
\item[API] Application Program Interface
\item[DART] Decoupled and Asynchronous Remote Transfers
\item[GTC] Gyrokinetic Turbulence Code
\item[HPC] High-Performance Computing
\item[I/O] Input/Output
\item[MDS] Metadata Server
\item[MPI] Message Passing Interface
\item[NCCS] National Center for Computational Sciences
\item[ORNL] Oak Ridge National Laboratory
\item[OS] Operating System
\item[PG] Process Group
\item[POSIX] Portable Operating System Interface
\item[RDMA] Remote Direct Memory Access
\item[XML] Extensible Markup Language
\end{description}


\vspace{18pt}
\begin{center}
{\large \textbf{Acknowledgments}}
\end{center}

\vspace{6pt}
The Adaptive Input/Output (I/O) system (ADIOS) is a joint product of the National 
Center of Computational Sciences (NCCS) at Oak Ridge National Laboratory (ORNL) 
and the Center for Experimental Research in Computer Systems at the Georgia Institute 
of Technology. This work is being led by Scott Klasky (ORNL); Jay Lofstead (Georgia 
Tech, funded from Sandia Labs) is the main contributor. ADIOS has greatly benefited 
from the efforts of the following ORNL staff: Steve Hodson, who gave tremendous 
input and guidance; Chen Jin, who integrated ADIOS routines into multiple scientific 
applications; Norbert Podhorszki, who integrated ADIOS with the Kepler workflow 
system and worked with Qing Gary Liu on the read API. ADIOS also benefited from 
the efforts of the Georgia Tech team, including Prof. Karsten Schwan, Prof. Matt 
Wolf, Hassan Abbasi, and Fang Zheng. Wei Keng Liao, Northwestern University, and 
Wang Di, SUN, have also been invaluable in our coding efforts of ADIOS, writing 
several important code parts. Essentially, ADIOS is componentization of I/O transport 
methods. Among the suite of transport methods, Decoupled and Asynchronous 
Remote Transfers (DART) was developed by Prof. Manish Parashar and his student 
Ciprian Docan of Rutgers University.

\vspace{6pt}
Without a scientific application, ADIOS would not have come this far. Special thanks 
go to Stephane Ethier at the Princeton Plasma Physics Laboratory (GTS); Researcher 
Yong Xiao and Prof. Zhihong Lin from the University of California, Irvine (GTC); 
Julian Cummings at the California Institute of Technology; Seung-Hoe and Prof. 
C. S. Chang at New York University (XGC); Jackie Chen and Ray Grout at Sandia (S3D); 
and Luis Chacon at ORNL (Pixie3D). 

\vspace{6pt}
This project is sponsored by ORNL, Georgia Tech, The Scientific Data Management 
Center (SDM) at Lawrence Berkeley National Laboratory, and the U.S. Department 
of Defense. 

%\vspace{24pt}
%\pagebreak
%{\Large \textbf{ADIOS contributors}}
%
%\vspace{6pt}
%\begin{description}
%\item[ANL]: Rob Ross
%\item[Auburn University]: Weikuan Yu, Yuan Tian
%\item[Georgia Tech]: Hasan Abbasi, Jay Lofstead, Karsten Schwan, Fang Zheng,
%\item[NCSU]: Xiaosong Ma, Sriram Lakshminarasimhan, Abhijit Sachidananda, Michael Warren 
%\item[Northwestern University]: Alok Choudhary, Wei Keng Liao, Chen Jin
%\item[ORNL]: Steve Hodson, Scott Klasky, Qing Gary Liu, Jeremy Logan,
%Norbert Podhorszki, Steve Poole, Nagiza Samatova, Matthew Wolf
%\item[Sandia]: Todd Kordenbrock
%\item[SUN]: Wang Di
%\end{description}

