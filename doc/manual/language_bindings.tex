\chapter{Language bindings}

ADIOS provides the following wrappers to support various programming environments; 
\begin{itemize}
\item{\bf Java} -- Write and Read ADIOS-BP files, with old read API
\item{\bf Python/Numpy} -- Write and Read ADIOS-BP files, with old read API
%\item{\bf Matlab} Read ADIOS-BP files, with own matlab/ADIOS commands
\end{itemize}

In this chapter, we will describe how one can use ADIOS wrappers and provide a few example codes.

\section{Java support}
\label{section-bindings-java}
The Java wrapper program consists of a set of Java classes defined with a single namespace, \verb+gov.ornl.ccs+. A list of classes is as follows:
\begin{itemize}
\item{\bf Adios} -- Provides functions to call init/free, write, and no-XML related APIs. All functions are static.
\item{\bf AdiosFile} -- Related with Read APIs. Represents ADIOS file structure.
\item{\bf AdiosGroup} -- Related with Read APIs. Represents ADIOS group structure. 
\item{\bf AdiosVarinfo} -- Related with Read APIs. Represents ADIOS varinfo structure.
\item{\bf AdiosDatatype} -- Enumeration class for ADIOS data types.
\item{\bf AdiosFlag} -- Enumeration class for ADIOS flags.
\item{\bf AdiosBufferAllocWhen} -- Enumeration class for ADIOS buffer allocation flags.
\end{itemize}

\subsection{Adios class}
This class provides static functions for initialization, finalization, writing, and no-XML related APIs. The list of functions and signatures are as follows:
\begin{lstlisting}[language=Java,caption={Member functions in the Adios class},label={}]
    /* Call adios_init */
    public static int Init(String xml_fname)

    /* Call adios_open. Return a group handler */
    public static long Open(String group_name, String file_name, 
                            String mode, long comm)

    /* Call adios_group_size and return the total size */
    public static long SetGroupSize(long fh, long group_size)

    /* Call adios_write and return the total size */
    public static long Write (long fh, String var_name, byte value)
    public static long Write (long fh, String var_name, int value)
    public static long Write (long fh, String var_name, long value)
    public static long Write (long fh, String var_name, float value)
    public static long Write (long fh, String var_name, double value)
    public static long Write (long fh, String var_name, byte[] value)
    public static long Write (long fh, String var_name, int[] value)
    public static long Write (long fh, String var_name, long[] value)
    public static long Write (long fh, String var_name, float[] value)
    public static long Write (long fh, String var_name, double[] value)

    /* Call adios_close */
    public static int Close (long fh)

    /* Call adios_finalize */
    public static int Finalize (int id)

    /* Call MPI_Init */
    public static int MPI_Init(String[] args)

    /* Call MPI_Comm_rank */
    public static int MPI_Comm_rank(long comm)

    /* Call MPI_Comm_size */
    public static int MPI_Comm_size(long comm)

    /* Call MPI_Finalize */
    public static int MPI_Finalize()

    /* Get MPI_COMM_WORLD */
    public static long MPI_COMM_WORLD()

    /* Call adios_init_noxml */
    public static int Init_Noxml()

    /* Call adios_allocate_buffer */
    public static int AllocateBuffer(AdiosBufferAllocWhen when, long size)

    /* Call adios_declare_group */
    public static long DeclareGroup(String name, String time_index, 
                                    AdiosFlag stats)

    /* Call adios_define_var */
    public static int DefineVar(long group_id, String name, String path, 
                                AdiosDatatype type, String dimensions, 
                                String global_dimensions, 
                                String local_offsets)

    /* Call adios_define_attribute */
    public static int DefineAttribute(long group_id, String name, 
                                String path, AdiosDatatype type, 
                                String value, String var)

    /* Call adios_select_method */
    public static int SelectMethod(long group_id, String method, 
                                String parameters, String base_path)
\end{lstlisting}

\subsection{AdiosFile, AdiosGroup, and AdiosVarinfo classes}
AdiosFile, AdiosGroup, and AdiosVarinfo classes represent \verb+ADIOS_FILE+, \verb+ADIOS_GROUP+, \verb+ADIOS_VARINFO+ structure, respectively, defined in \verb+adios_read_v1.h+. The following is a skeletal descriptions of those classes and member functions.

\begin{lstlisting}[language=Java,caption={Class definitions of AdiosFile, AdiosGroup, and AdiosVarinfo},label={}]
public class AdiosFile
{
    /* Call adios_fopen */
    public int open(String path, long comm)
    
    /* Call adios_fclose */
    public int close()
    
    /* Print contents for debugging purpose */
    public String toString()
}

public class AdiosGroup
{
    /* Constructor. Need AdiosFile instance */
    public AdiosGroup(AdiosFile file)
    
    /* Call adios_gopen */
    public int open(String grpname)
    
    /* Call adios_gclose */
    public int close()
    
    /* Print contents for debugging purpose */
    public String toString()
}

public class AdiosVarinfo
{
    /* Constructor. Need AdiosGroup instance */
    public AdiosVarinfo(AdiosGroup group)
    
    /* Call adios_inq_var */
    public int inq(String varname)
    
    /* Call adios_free_varinfo */
    public int close()
    
    /* Call adios_read_var */
    public double[] read(long[] start, long[] count)
    
    /* Print contents for debugging purpose */
    public String toString()
}
\end{lstlisting}

\subsection{AdiosDatatype, AdiosFlag, and AdiosBufferAllocWhen classes}
AdiosDatatype, AdiosFlag, and AdiosBufferAllocWhen are enumeration classes representing \verb+ADIOS_DATATYPES+, \verb+ADIOS_FLAG+, \verb+ADIOS_BUFFER_ALLOC_WHEN+ enum type, respectively, defined in \verb+adios_types.h+. The following is a skeletal descriptions of those classes and member functions.
\begin{lstlisting}[language=Java,caption={Enum classes},label={}]
public enum AdiosDatatype {
    UNKNOWN(-1),             /* (SIZE) */
    BYTE(0),                 /* (1) */
    SHORT(1),                /* (2) */
    INTEGER(2),              /* (4) */
    LONG(4),                 /* (8) */
    
    UNSIGNED_BYTE(50),       /* (1) */
    UNSIGNED_SHORT(51),      /* (2) */
    UNSIGNED_INTEGER(52),    /* (4) */
    UNSIGNED_LONG(54),       /* (8) */
    
    REAL(5),                 /* (4) */
    DOUBLE(6),               /* (8) */
    LONG_DOUBLE(7),          /* (16) */
    
    STRING(9),               /* (?) */
    COMPLEX(10),             /* (8) */
    DOUBLE_COMPLEX(11);      /* (16) */
}

public enum AdiosFlag {
    UNKNOWN(0),
    YES(1), 
    NO(2);
}

public enum AdiosBufferAllocWhen {
    UNKNOWN(0),
    NOW(1), 
    LATER(2);

}
\end{lstlisting}


\subsection{Example}
An example of Java program to call ADIOS functions is as follows:
\begin{lstlisting}[language=Java,caption={Example Java wrapper code},label={}]
import gov.ornl.ccs.*;
import java.nio.ByteBuffer;

public class AdiosNoxmlExample
{
    // The main program
    public static void main(String[] args)
    {
        Adios.MPI_Init(new String[0]);
        long comm = Adios.MPI_COMM_WORLD();
        int rank = Adios.MPI_Comm_rank(comm);
        int size = Adios.MPI_Comm_size(comm);

        Adios.Init_Noxml();
        Adios.AllocateBuffer(AdiosBufferAllocWhen.NOW, 10);

        long group_id = Adios.DeclareGroup("restart", "iter", 
                                           AdiosFlag.YES);
        Adios.SelectMethod(group_id, "MPI", "", "");
        Adios.DefineVar(group_id, "NX", "", 
                        AdiosDatatype.INTEGER, "", "", "");
        Adios.DefineVar(group_id, "G", "", 
                        AdiosDatatype.INTEGER, "", "", "");
        Adios.DefineVar(group_id, "O", "", 
                        AdiosDatatype.INTEGER, "", "", "");
        Adios.DefineVar(group_id, "temperature", "", 
                        AdiosDatatype.DOUBLE, "NX", "G", "O");

        long adios_handle = Adios.Open("restart", "adios_noxml.bp", 
                                       "w", comm);

        int NX = 10; 
        int G = NX * size;
        int O = NX * rank;

        double[] t = new double[NX];
        for (int i = 0; i < NX; i++) {
            t[i] = rank * NX + (double) i;
        }

        long groupsize = 4 + 4 + 4 + 8 * (1) * (NX);
        
        long adios_totalsize = Adios.SetGroupSize(adios_handle, groupsize);
        
        Adios.Write (adios_handle, "NX", NX);
        Adios.Write (adios_handle, "G", G);
        Adios.Write (adios_handle, "O", O);
        Adios.Write (adios_handle, "temperature", t);
        Adios.Close (adios_handle);
        
        Adios.Finalize (rank);        
        Adios.MPI_Finalize();
    }
}
\end{lstlisting}

\section{Python/Numpy support}
\label{section-bindings-numpy}
We developed a ADIOS python wrapper by using Cython. Numpy, a scientific module for Python, is a mandatory requirement. 

\subsection{APIs for Writing and No-XML}
The ADIOS python/numpy wrapper provides functions to call ADIOS write and no-XML related APIs as follows (defined in Cython syntax):
\begin{lstlisting}[language=cython,caption={Functions for writing and No-XML},label={},]
""" Call adios_init """
cpdef init(char * config)

""" Call adios_open """
cpdef int64_t open(char * group_name,
                   char * name,
                   char * mode,
                   MPI.Comm comm = MPI.COMM_WORLD)

""" Call adios_group_size """
cpdef int64_t set_group_size(int64_t fd_p, uint64_t data_size)

""" Call adios_write """
cpdef int write (int64_t fd_p, char * name, np.ndarray val)
cpdef int write_int (int64_t fd_p, char * name, int val)
cpdef int write_long (int64_t fd_p, char * name, long val)
cpdef int write_float (int64_t fd_p, char * name, float val)

""" Call adios_read """
cpdef int read(int64_t fd_p, char * name, np.ndarray val)

""" Call adios_close """
cpdef int close(int64_t fd_p)

""" Call adios_finalize """
cpdef finalize(int mype = 0)

""" Call adios_init """
cpdef init(char * config)

""" Call adios_open """
cpdef int64_t open(char * group_name,
                   char * name,
                   char * mode,
                   MPI.Comm comm = MPI.COMM_WORLD)

""" Call adios_group_size """
cpdef int64_t set_group_size(int64_t fd_p, uint64_t data_size)

""" Call adios_write """
cpdef int write (int64_t fd_p, char * name, np.ndarray val)
cpdef int write_int (int64_t fd_p, char * name, int val)
cpdef int write_long (int64_t fd_p, char * name, long val)
cpdef int write_float (int64_t fd_p, char * name, float val)

""" Call adios_read """
cpdef int read(int64_t fd_p, char * name, np.ndarray val)

""" Call adios_close """
cpdef int close(int64_t fd_p)

""" Call adios_finalize """
cpdef finalize(int mype = 0)

""" Call adios_init_noxml """
cpdef int init_noxml():

""" Call adios_allocate_buffer """
cpdef int allocate_buffer(int when,
                          uint64_t buffer_size)

""" Call adios_declare_group """
cpdef int64_t declare_group(char * name,
                            char * time_index,
                            int stats)

""" Call adios_define_var """
cpdef int define_var(int64_t group_id,
                     char * name,
                     char * path,
                     int type,
                     char * dimensions,
                     char * global_dimensions,
                     char * local_offsets)

""" Call adios_define_attribute """
cpdef int define_attribute (int64_t group,
                            char * name,
                            char * path,
                            int type,
                            char * value,
                            char * var)

""" Call adios_select_method """
cpdef int select_method (int64_t group,
                         char * method,
                         char * parameters,
                         char * base_path)
\end{lstlisting}

\subsection{APIs for Reading}
The ADIOS python/numpy wrapper provides ADIOS read related classes as follows (defined in Cython syntax):
\begin{lstlisting}[language=cython,caption={Write functions},label={},]
""" Python class for ADIOS_FILE structure """
cdef class AdiosFile:
    """ Private Memeber """
    cpdef ADIOS_FILE * fp

    """ Public Memeber """
    cpdef public bytes name
    cpdef public int groups_count
    cpdef public int vars_count
    cpdef public int attrs_count
    cpdef public int tidx_start
    cpdef public int ntimesteps
    cpdef public int version
    cpdef public int file_size
    cpdef public int endianness
    
    cpdef public dict group
    
    """ Initialization. Call adios_fopen and populate public members """
    def __init__(self, char * fname, MPI.Comm comm = MPI.COMM_WORLD):
        ...
    
    """ Call adios_fclose """
    cpdef close(self):
        ...
        
    """ Print self """
    cpdef printself(self):
        ...

""" Python class for ADIOS_GROUP structure """
cdef class AdiosGroup:
    """ Private Memeber """
    cdef AdiosFile file
    cdef ADIOS_GROUP * gp

    """ Public Memeber """
    cpdef public bytes name
    cpdef public int grpid
    cpdef public int vars_count
    cpdef public int attrs_count
    cpdef public int timestep
    cpdef public int lasttimestep
    
    cpdef public dict var
    
    """ Initialization. Call adios_gopen and populate public members """
    def __init__(self, AdiosFile file, char * name):

    """ Call adios_gclose """
    cpdef close(self):
        ...
        
    """ Print self """
    cpdef printself(self):
        ...

""" Python class for ADIOS_VARINFO structure """
cdef class AdiosVariable:
    """ Private Memeber """
    cdef AdiosGroup group
    cdef ADIOS_VARINFO * vp

    """ Public Memeber """
    cpdef public bytes name
    cpdef public int varid
    cpdef public type type
    cpdef public int ndim
    cpdef public tuple dims
    cpdef public int timedim
    cpdef public int characteristics_count
    
    """ Initialization. Call adios_inq_var and populate public members """
    def __init__(self, AdiosGroup group, char * name):
        ...
        
    """ Call adios_free_varinfo """
    cpdef close(self):
        ...
        
    """ Call adios_read_var """
    cpdef read(self, tuple offset = (), tuple count = ()):
        ...

    """ Print self """
    cpdef printself(self):
        ...

\end{lstlisting}

\subsection{Example}
An example of Python program to call ADIOS functions is shown below. The example is a Python program for converting a NetCDF file to a ADIOS BP file. You can find the code in the source distribution: \verb+/wrapper/numpy/example/ncdf2bp.py+.

\begin{lstlisting}[language=ADIOS-python,caption={ncdf2bp.py. An example Python/Numpy wrapper code for converting a NetCDF file to a ADIOS BP file},label={},]
#!/usr/bin/env python
from adios import *
from scipy.io import netcdf
import numpy as np
import sys
import os
import operator

def usage():
    print os.path.basename(sys.argv[0]),"netcdf_file","[time dimension name]"

if len(sys.argv) < 2:
    usage()
    sys.exit(0)

fname = sys.argv[1]
fout = '.'.join(fname.split('.')[:-1]) + ".bp"

tname = "time"
if len(sys.argv) > 2:
    tname = sys.argv[2]

## Open NetCDF file
f = netcdf.netcdf_file(fname, 'r')

## Check dimension
assert (all(map(lambda x: x is not None,
                [ val for k, val in f.dimensions.items()
                  if k != tname])))

## Two types of variables : time-dependent or time-independent
dimvar = {n:v for n,v in f.variables.items() if n in f.dimensions.keys()}
var = {n:v for n,v in f.variables.items() if n not in f.dimensions.keys()}
tdepvar = {n:v for n,v in var.items() if tname in v.dimensions}
tindvar = {n:v for n,v in var.items() if tname not in v.dimensions}

## Time dimension
assert (len(set([v.dimensions.index(tname) for v in tdepvar.values()]))==1)
tdx = tdepvar.values()[0].dimensions.index(tname)

assert (all([v.data.shape[tdx] for v in tdepvar.values()]))
tdim = tdepvar.values()[0].shape[tdx]

## Init ADIOS without xml
init_noxml()
allocate_buffer(BUFFER_ALLOC_WHEN.NOW, 10)
gid = declare_group ("group", tname, FLAG.YES)
select_method (gid, "MPI", "", "")

d1size = 0
for name, val in f.dimensions.items():
    if name == tname:
        continue
    print "Dimension : %s (%d)" % (name, val)
    define_var (gid, name, "", DATATYPE.integer, "", "", "")
    d1size += 4
    
v2size = 0
for name, var in tdepvar.items():
    print "Variable : %s (%s)" % (name, ','.join(var.dimensions))
    define_var (gid, name, "", DATATYPE.double,
                ','.join(var.dimensions),
                ','.join([dname for dname in var.dimensions
                          if dname != tname]),
                "0,0,0")
    v2size += reduce(operator.mul, var.shape) / tdim * 8

print "Count (dim, var) : ", (d1size, v2size)

## Clean old file
if os.access(fout, os.F_OK):
    os.remove(fout)

for it in range(tdim):
    print 
    print "Time step : %d" % (it)
    
    fd = open("group", fout, "a")
    groupsize = d1size + v2size
    set_group_size(fd, groupsize)

    for name, val in f.dimensions.items():
        if name == tname:
            continue
        print "Dimension writing : %s (%d)" % (name, val)
        write_int(fd, name, val)
        
    for name, var in tdepvar.items():
        arr = np.array(var.data.take([it], axis=tdx),
                       dtype=np.float64)
        print "Variable writing : %s %s" % (name, arr.shape)
        write(fd, name, arr)

    close(fd)
    
f.close()
finalize()

print
print "Done. Saved:", fout
    
\end{lstlisting}


%\section{Matlab support}
%\label{section-bindings-matlab}

