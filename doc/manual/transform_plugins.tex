\chapter{Data Transformations}
\label{sec:transform_plugins}
The ADIOS data transformations framework, new in version 1.6, enables on-the-fly, transparent ``data transformations'' during the ADIOS
write phase. Data transformations are a class of algorithms that change the format of a variable's data for some purpose
(for example, compression to reduce storage footprint).
Data transforms in ADIOS are fully configurable on a per-variable basis; for instance, one variable
may be compressed while another has no transform applied.

While ADIOS comes with lossless compression transform plugins, it is possible for plugin developers to implement new plugins.
Possibilities include: lossy compression, layout optimization for increased spatial locality,
on-the-fly value indexing, and precision-based and sampling-based level-of-detail encodings.

It is important to note that data transforms are \emph{runtime configurable}, meaning selecting a new configuration of data
transforms for variables in the ADIOS XML does \emph{not} require recompiling the application (analogously to I/O transport
selection via the XML). Furthermore, beyond editing the XML, no other changes are required to application code to use data
transforms during writing or reading. This makes it possible to easily experiment with the effects of a variety of compression
methods and parameters in an application.

\section{Available data transformations}
The \verb+adios_config+ tool lists the transformation methods (as well as write/read methods) when using the -m option.

\begin{lstlisting}
$ adios_config -m
...
Available data transformation methods (in XML transform tags in <var> elements):
    "none"     : No data transform
    "identity" : Identity transform
    "bzip2"    : bzip2 compression
    "isobar"   : ISOBAR compression
    "alacrity" : ALACRITY indexing
    "zfp"      : zfp compression
\end{lstlisting}


\section{Writing with data transformations}
Data transforms are selected via the ADIOS XML file, as follows:

\begin{lstlisting}[language=XML]
<var name="/temperature"
     dimensions="..."
     type="adios_real"
     transform="zlib"    <!-- Add this attribute -->
  />
\end{lstlisting}

The above snippet of XML will cause the variable "/temperature" to be compressed using zlib, assuming ADIOS was configured with zlib support
(see Section~\ref{sec:installation-data-transforms} for configuration instructions). \adiosversion includes support for 
various compression (lossy and lossless) and indexing transform plugins (see Table~\ref{tbl:data-transforms-summary}).
Some transform plugins accept parameters, which are appended after a colon following the transform name.
For instance, zlib accepts an optional compression level between 1 and 9, with 1 being the fastest and
9 being the highest compression ratio. Setting the level to 5 would be accomplished as follows:

\begin{lstlisting}[language=XML]
<var name="/temperature"
     ...
     transform="zlib:5"/>
\end{lstlisting}

In fact, all three lossless compression plugins (zlib, bzip2, and szip) currently accept this same
1-to-9 compression level. The default compression for each library is used if this parameter is
omitted, which is typically the case.

The zfp transform accommodates the three high-level compression modes of the zfp API: rate, precision, and accuracy.
In the compression settings string, users specify one the compression modes followed by an \texttt{=} and a numerical value.
We refer users to \href{http://computation.llnl.gov/projects/floating-point-compression}{zfp software's} README for details,
but exemplify the three possibilities below:
\begin{lstlisting}[language=XML]
<var name="/temperature"
     ...
     transform="zfp:rate=0.25"
     transform="zfp:precision=16"
     transform="zfp:accuracy=0.0001"
     />
\end{lstlisting}
Rate means a constant number of bits will be kept per zfp-block, where the rate is $ kept \; bits/ (64*blocksize)$ for 64-bit doubles.
The precision value is an integer, which sets a maximum number of bit planes to output (up to a max of 64 for floating point doubles) -- 
this is equivalent to setting a relative precision.
Accuracy specifies an absolute error toleranece.


\begin{table}%
\begin{tabular}{l|l|l}
\textbf{Transform name in XML} & \textbf{Description} & \textbf{Other info} \\
\hline
zlib & lossless compression & requires the zlib external library \\
\hline
bzip2 & lossless compression & requires the bzip2 external library \\
\hline
szip & lossless compression & requires the szip external library \\
\hline
zfp & lossy compression & included with ADIOS \\
\hline
isobar & lossless compression & requires the isobar external library \\
\hline
alacrity & indexing for queries & requires the alacrity external library \\
\end{tabular}
\caption{Summary of data transform plugins included in ADIOS}
\label{tbl:data-transforms-summary}
\end{table}

Data transformations that have been selected for a variable are applied at write time, during the call to \verb+adios_write()+.
Data transforms are applied on a per-core basis, so no additional data movement or other communication is incurred.
The resultant re-formatted data is then written using whatever I/O transport method is active.

\section{Reading with data transformations}
On the read side, no changes to application code are necessary. If a variable being read was transformed when written
(e.g., zlib compressed), the transformation will be inverted automatically, returning the original data,
with no changes needed to reader application.

\section{Considerations when selecting data transforms}
When deciding whether to apply data transforms, and which transform methods to use, depends on a few factors.
Since the built-in data transforms in ADIOS are compression routines, we will focus on this case, but much of
this discussion applies generally to any data transform.

First, because the transform is applied at write time, the user must consider the tradeoff between the CPU cost of applying
the transform (e.g., compression time) with the expected benefits (reduced storage space, and potentially reduced I/O time).
The balance point for a particular application can only be determined through experimentation, but for compression,
the more compressible the user's data is, the beneficial a compression transform may be.

Second, note that some transforms, including compression, necessarily make some read operations more costly. This is because
chunks of variable data in the file (the piece of the variable in each process group, to be precise) will be compressed
in its entirety, and so during reads, any process group chunk that is touched must be read and decompressed fully, whereas
without compression, some of the chunk might have been omitted from the read.

As before, experimentation is the only way to definitively identify application- and read-pattern-specific read overhead.
However, as a rule, smaller PG sizes and read access
patterns that access large parts of PGs (e.g., full timestep reads and subvolume selections, as opposed to plane cuts and
point selections) experience less overhead. For example, very large PG sizes (>10MB per variable per writing process per timestep)
combined with read patterns that touch small pieces of many PGs (such as plane cuts) can experience substantial overhead.
In contrast, checkpoint-restart-like access patterns that are highly aligned to PG boundaries,
this overhead should be negligible, and overall performance improvement is possible via reduced I/O time given a high enough compression
ratio.

Finally, note the compatibility limitations described in the next section.

\section{Compatibility}
As mentioned above, data transforms work seamlessly with existing read and write APIs; only the addition of the "transform" attribute
in the XML is required to invoke a data transform.

As for I/O transports, in this release only file-based transports are supported while data transforms are active.
This includes the \verb+POSIX+, \verb+MPI+, \verb+MPI_LUSTRE+, and \verb+MPI_AGGREGATE+ write transports and
the \verb+READ_BP+ read transport. Additionally,
when reading transformed data, only the file access mode is currently supported (i.e., streaming mode is unsupported).
Of course, none of these limitations apply to use of ADIOS when no data transforms is active
(i.e., no "transform" attributes in the XML for writes and no transformed variables are accessed for reads).

As a final note, due to the way in which space allocation occurs under compression data transforms,
applying compression in combination with certain I/O transports (including
\verb+MPI+ and \verb+MPI_LUSTRE+) will produce ``sparse files'' with ``holes'' (unused space).
While this may avoided entirely by using other transports (including \verb+MPI_AGGREGATE+ and
\verb+POSIX+) sparse files are not necessarily bad. Sparse files do not consume extra disk space or
I/O time on most filesystems (including Lustre, GPFS, PVFS and modern ext Linux filesystems),
and standard Linux file manipulation tools (including cp and rsync) usually handle such files efficiently.
Only if moved/copied using a tool (besides cp or rsync) that ``materializes'' them into normal ``dense files''
will they begin consume more space. A file can be tested for sparsenesss by comparing the output of ``du -h <file>'' and
``ls -l <file>'' (the former reports actual disk space used, whereas the latter reports logical file size;
for sparse files, these quantities differ significantly).
Again, sparse files may be completely avoided by using \verb+MPI_AGGREGATE+ or \verb+POSIX+.

As this is the first release of the ADIOS data transformation framework, optimization and improvements are ongoing, and
in the future many of these restrictions are likely to be ameliorated.
