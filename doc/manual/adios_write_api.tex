\chapter{ADIOS Write API}

As mentioned earlier, ADIOS writing is comprised of two parts: the XML configuration 
file and APIs. In this section, we will explain the functionality of the writing 
API in detail and how they are applied in the program.  

\section{Write API Description}

\subsection{Introduction}

ADIOS provides both Fortran and C routines. All ADIOS routines and constants begin 
with the prefix ``adios\_''. For the remainder of this section, only the C versions 
of ADIOS APIs are presented. The primary differences between the C and Fortran 
routines is that error codes are returned in a separate argument for Fortran as 
opposed to the return value for C routines. 

A unique feature of ADIOS is group implementation, which is constituted by a list 
of variables and associated with individual transport methods. This flexibility 
allows the applications to make the best use of the file system according to its 
own different I/O patterns.

\subsection{ADIOS-required functions}

This section contains the basic functions needed to integrate ADIOS into scientific 
applications. ADIOS is a lightweight I/O library, and there are only six required and an optional  
functions from which users can write scalable, portable programs with flexible 
I/O implementation on supported platforms:

\textbf{adios\_init ---} initialize ADIOS and load the configuration file

\textbf{adios\_open ---} open the group associated with the file

\textbf{adios\_group\_size ---} \textit{optional.} Pass the group size to allocate the memory for buffering

\textbf{adios\_write ---} write the data either to internal buffer or disk

\textbf{adios\_read ---} associate the buffer space for data read into

\textbf{adios\_close ---} commit write/read operation and close the data

\textbf{adios\_finalize ---} terminate ADIOS

You can add functions to your working knowledge incrementally without having to 
learn everything at once. For example, you can achieve better I/O performance on 
some platforms by simply adding the asynchronous functions adios\_start\_calculation, 
adios\_end\_calculation, and adios\_end\_iteration to your repertoire. These functions 
will be detailed below in addition to the seven indispensable functions.

The following provides the detailed descriptions of required APIs when users apply 
ADIOS in the Fortran or C applications.

\subsubsection{adios\_init}

This function is required only once during the program run. 
It loads the XML configuration file 
and establishes the execution environment. Before any ADIOS operation starts, adios\_init 
is required to be called to create internal representations of various data types 
and to define the transport methods used for writing. 
From version 1.5, this function does have an \verb+MPI_Comm comm+ 
argument. 

\begin{lstlisting}[alsolanguage=C]
int adios_init (const char * xml_fname, MPI_Comm comm)
\end{lstlisting}

Input: 
\begin{itemize}
\item xml\_fname - string containing the name of the XML configuration file
\item comm - MPI communicator. Any process that is going to use ADIOS should call this function and must be a member of this communicator. 
\end{itemize}

Fortran example: 
\begin{lstlisting}[alsolanguage=Fortran]
call adios_init ("config.xml", comm, ierr)
\end{lstlisting}

\subsubsection{adios\_open}

This function is to open or to append to an output file. adios\_open
opens an adios-group identified by \textit{group\_name}
and associates it with one or a list of transport methods. 
A pointer is returned as \textit{fd\_p} for subsequent operations. 
The group name should match one of the groups defined in the XML file. The I/O handle 
The third argument, \textit{file\_name}, is a string representing the name of 
the file. 
The fourth argument \textit{mode} is a string containing a file access mode. 
It can be one of these four mode specifiers: ``r,'' ``w,''  ``a.'' or ``u''. Currently, 
ADIOS supports four access modes: ``write or create if file does not exist,'' 
``read,'' ``append file,'' and ``update file.'' Appending increases the built-in timestep in the file, so when reading, the variables appear at different timesteps. Updating adds variables to the latest timestep in the file.
The last argument is the MPI communicator \textit{comm} that includes all
processes that write to the file. Individual writes can be called by individual processes, but \textit{adios\_group\_size} and \textit{adios\_close} are collective operations, that all processes under this communicator should call. 

Note, that a file is not necessarily opened during this call. Some methods postpone the actual file open to \textit{adios\_group\_size}.

Note, that before version 1.5, this function required a \verb+void * comm+ argument, so you need to update existing codes to pass the communicator itself now, not a pointer to it.

\begin{lstlisting}[alsolanguage=C]
int adios_open (int64_t * fd_p, const char * group_name,
            const char * file_name, const char * mode, MPI_Comm comm)
\end{lstlisting}

Input: 
\begin{itemize}
\item fd\_p---pointer to the internal file structure
\item group\_name---string containing the name of the group 
\item file\_name---string containing the name of the file to be opened 
\item mode---string containing  a file access mode
\item comm--- communicator for multi-process coordination
\end{itemize}

Fortran example: 
\begin{lstlisting}[alsolanguage=Fortran]
call adios_open (handle, "restart", "restart.bp", "w", comm, ierr)
\end{lstlisting}

\subsubsection{adios\_group\_size}
This optional function passes the size of the group to the internal ADIOS transport structure 
to facilitate the internal buffer management. ADIOS allocates a certain buffer in adios\_open()
for the data. During writing (adios\_write()), if there is more data is written than what the
buffer can hold, ADIOS tries to extend the buffer. This group size call can tell ADIOS how big buffer 
to allocate before the first write call. 

The first argument is the file handle. The second argument is the size of the payload 
(in bytes) for the group opened in the adios\_open routine that the specific process 
is going to write into the file. 
This value can be calculated manually, knowing the sizes of all variables to be written 
or through our python script \verb+gpp.py+ that generates and puts this calculation 
and \textit{adios\_group\_size} call along the write operations of all variables into 
a text file. It does not affect read operation because the size 
of the data can be retrieved from the file itself. The third argument is the returned 
value for the total size of this group, which is the payload size increased with the 
metadata overhead. The value can be used for performance benchmarks, such as I/O speed. 

Note that in the XML file, you can specify a maximum buffer size for ADIOS. The buffer is 
used to collect all outputs between one adios\_open -- adios\_close cycle, and all data
is written out during adios\_close. This maximizes the benefits of large I/O chunks and
is one of the key contributors of the superior performance of ADIOS file I/O. 
If the buffer is not large enough to hold all data for the output,
ADIOS methods either fall back to write data to the target in smaller chunks, that will
result in much worse write performance, or will simply not write more than what fits
in the buffer (an error message will be printed).

\begin{lstlisting}[alsolanguage=C]
int adios_group_size (int64_t * fd_p, uint64_t group_size, 
                      uint64_t * total_size)
\end{lstlisting}

Input: 
\begin{itemize}
\item fd\_p---pointer to the internal file structure
\item group\_size---size of data payload in bytes to be written out. If there is an integer 
2 $\times$ 3 array, the payload size is 4 $\times$ 2 $\times$ 3 (4 is the size of integer)
\end{itemize}

output :
\begin{itemize}
\item total\_size---the total sum of payload and overhead, which includes name, data 
type, dimensions and other metadata)
\end{itemize}

Fortran example: 
\begin{lstlisting}[alsolanguage=Fortran, caption={}]
call adios_group_size (handle, groupsize, totalsize, ierr)
\end{lstlisting}

\subsubsection{adios\_write}
The adios\_write routine submits a data element var for writing and associates 
it with the given var\_name, which has been defined in the XML definition of the 
corresponding adios group opened by 
adios\_open. If the ADIOS buffer is big enough to hold all the data that the adios 
group needs to write, this API only copies the data to buffer. Otherwise, adios\_write 
will write to disk without buffering. When the function returns, the memory pointed
by \textit{var} can be reused by the application. 
Adios\_write expects the address of the contiguous block of memory to be written. 
A noncontiguous array, comprising a series of subcontiguous memory blocks, 
should be given separately for each piece.

In the next chapter about the XML file, we will further explain that the var\_name
argument of this function should correspond to combined path value of the
attributes ``path'' and ``name'' in the variable definition. 
Another attribute, ``gwrite,'' is used by \verb+gpp.py+ to generate the variable name in
the application source code, that is passed as \textit{var} in this call. 
See the \verb+<var>+ element inside the \verb+<adios_group>+ element 
in the XML file. 
If ``gwrite'' is not defined, it will be handled as if it were the same as the value of attribute ``name''.

Since version 1.6, the matching of variable names in the XML and this function call 
is strict. If the ``path'' attribute is used in the XML, simply referring to the 
``name'' attribute in the function call will fail. One need to use the full path of 
the variable at both writing and reading. 

\begin{lstlisting}[alsolanguage=C,caption={},label={}]
int adios_write (int64_t fd_p, const char * var_name, const void * var)
\end{lstlisting}

Input:
\begin{itemize}
\item fd\_p---pointer to the internal file structure
\item var\_name---string containing the annotation name of scalar or vector in the XML file
\item var ---the address of the data element defined need to be written
\end{itemize}

Fortran example: 
\begin{lstlisting}[alsolanguage=Fortran,caption={},label={}]
call adios_write (handle, "myvar", v, ierr)
\end{lstlisting}

\subsubsection{adios\_read}
\label{section:adios_read}

{\bf Obsolete function}.
The write API contains a read function (historically, the first one) that uses 
the same transport method and the xml config file to read in data. It works only 
on the same number of processes as the data was written out. Typically, checkpoint/restart 
files are written and read on the same number of processors and this function is 
the simplest way to read in data. However, if you need to read in on a different 
number of processors, use a transport method that does not support read (e.g. 
the MPI\_AGGREGATE method) or you do not want to carry the xml config file with the 
reading application, you should use the newer and more generic read API discussed 
in Section 7.

Similar to adios\_write, adios\_read passes the buffer space in the \textit{var} argument 
for reading a data element into. This does NOT actually perform the read. Actual 
population of the buffer space will happen on the call to adios\_close. In other 
words, the value(s) of var can only be utilized after adios\_close is performed. 
Here, var\_name corresponds to the value of attribute ``gread`` in the \verb+<var>+
element declaration while var is mapped to the value of attribute ``name.'' By 
default, it will be as same as the value of attribute ``name'' if ``gread'' is 
not defined.

\begin{lstlisting}[alsolanguage=C,caption={},label={}]
int adios_read (int64_t fd_p, const char * var_name, 
                uint64_t read_size, void * var)
\end{lstlisting}

Input:

\begin{itemize}
\item fd\_p - pointer to the internal file structure
\item var\_name - the name of variable recorded in the file
\item var - the address of variable defined in source code
\item read\_size -  size in bytes of the data to be read in 
\end{itemize}

Fortran example: 
\begin{lstlisting}[alsolanguage=Fortran,caption={},label={}]
call adios_read (handle, "myvar", 8, v, ierr)
\end{lstlisting}

\subsubsection{adios\_close}

The adios\_close routine commits the writing buffer to disk, closes 
the file, and releases the handle. At that point, all of the data that have been 
copied during adios\_write will be sent as-is downstream. If the file was opened 
for read, this function fetches all data and populates it into the 
buffers provided in the adios\_read calls. 

\begin{lstlisting}[alsolanguage=C,caption={},label={}]
int adios_close (int64_t fd_p);
\end{lstlisting}

Input: 
\begin{itemize}
\item fd\_p - pointer to the internal file structure
\end{itemize}

Fortran example: 
\begin{lstlisting}[alsolanguage=Fortran,caption={},label={}]
call adios_close (handle, ierr)
\end{lstlisting}

\subsubsection{adios\_finalize}

The adios\_finalize routine releases all the resources allocated by ADIOS and guarantees 
that all remaining ADIOS operations are finished before the code exits. The ADIOS 
execution environment is terminated once the routine is fulfilled. The proc\_id 
parameter provides developers of ADIOS transport methods the opportunity to customize 
some special operations based on the proc\_id---usually on one process. 

\begin{lstlisting}[alsolanguage=C,caption={},label={}]
int adios_finalize (int proc_id)
\end{lstlisting}

Input: 
\begin{itemize}
\item proc\_id - the rank of the process (in the MPI application)  
\end{itemize}

Fortran example: 
\begin{lstlisting}[alsolanguage=Fortran,caption={},label={}]
call adios_finalize (rank, ierr)
\end{lstlisting}

call adios\_finalize (rank, ierr)

\subsection{Asynchronous I/O support functions}

\subsubsection{adios\_end\_iteration}

The adios\_end\_iteration provides the pacing indicator. Based on the entry in 
the XML file, it will tell the transport method how much time has elapsed in a 
transfer. Applications usually perform computation in an iterative loop, and write
data with a regular frequency (but not at every iteration). This function, if called
at each iteration, can provide hints to the ADIOS layer about the progress of the 
application and thus estimate the remaining time to the next output phase. Asynchronous
I/O methods can use this estimate to trickle the data of the previous output as slow
as possible to minimize interference with the application. 

\subsubsection{adios\_start\_ calculation/ adios\_end\_calculation}

Together, adios\_start\_calculation and adios\_end\_calculation indicate  
to asynchronous methods when they should focus on engaging their I/O communication 
efforts because the process is mainly performing intense, stand-alone computation. 
Otherwise, the code is deemed likely to be communicating heavily for computation 
coordination. Any attempts to write or read during collective communication of 
the application will negatively 
impact both the asynchronous I/O performance and the interprocess messaging.

\subsection{Other functions}

One of our design goals is to keep ADIOS APIs as simple as possible. In addition 
to the basic I/O functions, we provide another routine listed below. 

%\subsubsection{adios\_get\_write\_buffer}
%
%  NOTE: We have no idea what this function is for, if anyone ever used it 
%        and how it works in the actual methods.
%
%
%The adios\_get\_write\_buffer function returns the buffer space allocated from 
%the ADIOS buffer domain. In other words, instead of allocating memory from free 
%memory space, users can directly use the allocated ADIOS buffer area and thus avoid 
%copying memory from the ADIOS buffer to a user-defined buffer.
%
%\begin{lstlisting}[alsolanguage=C,caption={},label={}]
%int adios_get_write_buffer (int64_t fd_p, const char * var_name, 
%    uint64_t * size, void ** buffer)
%\end{lstlisting}
%
%Input: 
%\begin{itemize}
%\item fd\_p - pointer to the internal File structure
%\item var\_name - name of the variable that will be read
%\item size - size of the buffer to request
%\end{itemize}
%
%Output: 
%\begin{itemize}
%\item buffer - initial address of read-in buffer for storing the data of var\_name
%\end{itemize}


\section{Write Fortran API description}
\label{section:write_fortran_api}

A Fortran90 module, \verb+adios_write_mod.mod+ provides the ADIOS write 
subroutines discussed above. 
They are all interfaced to the C library. Their extra last argument 
(compared to the corresponding C functions) is an integer variable to store 
the error code output of each function (0 meaning successful operation). 

Here is the list of the Fortran90 subroutines from \verb+adios_write_mod.mod+. 
In the list below \verb+GENERIC+ word indicates that you 
can use that function with any data type at the indicated argument; it is not
a Fortran90 keyword. The actual module source defines all possible combinations 
of type and dimensionality for such subroutines. For attribute definition, \verb+GENERIC+ 
stands for a scalar value or a 1-dimensional array of any type. 

\begin{lstlisting}[language=ADIOS,alsolanguage=Fortran]
subroutine adios_init (config, comm, err)
    character(*),   intent(in)  :: config
    integer,        intent(in)  :: comm
    integer,        intent(out) :: err
end subroutine

subroutine adios_init_noxml (comm, err)
    integer,        intent(in)  :: comm
    integer,        intent(out) :: err
end subroutine

subroutine adios_finalize (mype, err)
    integer,        intent(in)  :: mype
    integer,        intent(out) :: err
end subroutine

subroutine adios_open (fd, group_name, filename, mode, comm, err)
    integer*8,      intent(out) :: fd
    character(*),   intent(in)  :: group_name
    character(*),   intent(in)  :: filename
    character(*),   intent(in)  :: mode
    integer,        intent(in)  :: comm
    integer,        intent(out) :: err
end subroutine

subroutine adios_group_size (fd, data_size, total_size, err)
    integer*8,      intent(out) :: fd
    integer*8,      intent(in)  :: data_size
    integer*8,      intent(in)  :: total_size
    integer,        intent(out) :: err
end subroutine

subroutine adios_write (fd, varname, data, err)
    integer*8,      intent(in)  :: fd
    character(*),   intent(in)  :: varname
    GENERIC,        intent(in)  :: data
    integer,        intent(in)  :: err 
end subroutine

subroutine adios_read (fd, varname, buffer, buffer_size, err)
    integer*8,      intent(in)  :: fd
    character(*),   intent(in)  :: varname
    GENERIC,        intent(out) :: buffer
    integer*8,      intent(in)  :: buffer_size 
    integer,        intent(in)  :: err 
end subroutine

subroutine adios_set_path (fd, path, err)
    integer*8,      intent(in)  :: fd
    character(*),   intent(in)  :: path
    integer,        intent(out) :: err
end subroutine

subroutine adios_set_path_var (fd, path, varname, err)
    integer*8,      intent(in)  :: fd
    character(*),   intent(in)  :: path
    character(*),   intent(in)  :: varname
    integer,        intent(out) :: err
end subroutine

subroutine adios_end_iteration (err)
    integer,        intent(out) :: err
end subroutine

subroutine adios_start_calculation (err)
    integer,        intent(out) :: err
end subroutine

subroutine adios_stop_calculation (err)
    integer,        intent(out) :: err
end subroutine

subroutine adios_close (fd, err)
    integer*8,      intent(in)  :: fd
    integer,        intent(out) :: err
end subroutine

!
! No-XML calls
!
subroutine adios_declare_group (id, groupname, time_index, stats_flag, err)
    integer*8,      intent(out) :: id
    character(*),   intent(in)  :: groupname
    character(*),   intent(in)  :: time_index
    integer,        intent(in)  :: stats_flag
    integer,        intent(out) :: err
end subroutine

subroutine adios_define_var (group_id, varname, path, vartype, dimensions, global_dimensions, local_offsets, err)
    integer*8,      intent(in)  :: group_id
    character(*),   intent(in)  :: varname
    character(*),   intent(in)  :: path
    integer,        intent(in)  :: vartype
    character(*),   intent(in)  :: dimensions
    character(*),   intent(in)  :: global_dimensions
    character(*),   intent(in)  :: local_offsets
    integer*8,      intent(out) :: id
end subroutine

subroutine adios_define_attribute (group_id, attrname, path, attrtype, value, varname, err)
    integer*8,      intent(in)  :: group_id
    character(*),   intent(in)  :: attrname
    character(*),   intent(in)  :: path
    integer,        intent(in)  :: attrtype
    character(*),   intent(in)  :: value
    character(*),   intent(in)  :: varname
    integer,        intent(out) :: err
end subroutine

subroutine adios_define_attribute_byvalue (group_id, attrname, path, nelems, values, err)
    integer*8,      intent(in)  :: group_id
    character(*),   intent(in)  :: attrname
    character(*),   intent(in)  :: path
    integer,        intent(in)  :: nelems
    GENERIC,   intent(in)  :: values
    integer,        intent(out) :: err
end subroutine

subroutine adios_select_method (group_id, method, parameters, base_path, err)
    integer*8,      intent(in)  :: group_id
    character(*),   intent(in)  :: method
    character(*),   intent(in)  :: parameters
    character(*),   intent(in)  :: base_path
    integer,        intent(out) :: err
end subroutine

subroutine adios_set_max_buffer_size (sizeMB)
    implicit none
    integer,        intent(in)  :: sizeMB
end subroutine
\end{lstlisting}

\subsection{Create the first ADIOS program}

Listing \ref{list-adios-prog-example} is a programming example that illustrates 
how to write a double-precision 
array \verb+t+ of size of \verb+NX+ into file called ``test.bp,'' 
which is organized in BP, our native tagged binary file format. This format allows 
users to include rich metadata associated with the block of binary data as well 
the indexing mechanism for different blocks of data (see Chapter \ref{chapter-xml}). 

\begin{lstlisting}[alsolanguage=C,caption={ADIOS programming example.},label={list-adios-prog-example}]
/*example of parallel MPI write into a single file */ 
#include <stdio.h> // ADIOS header file required 
#include "adios.h"
int main (int argc, char *argv[])
{    
    int i, rank; 
    int NX = 10;
    double t [NX];
    // ADIOS variables declaration int64_t handle;
    uint_64 group_size, total_size;
    MPI_Comm comm = MPI_COMM_WORLD; 
    
    MPI_Init ( &argc, &argv);
    MPI_Comm_rank (comm, &rank);

    // data initialization for ( i=0; i<NX; i++)
    t [i] = i * (rank+1) + 0.1; // ADIOS routines 
    adios_init ("config.xml", comm);
    adios_open (&handle, "temperature", "data.bp", "w",comm); 
    group_size = sizeof(int) \        // int NX
               + sizeof(double) * NX; // double array t
    adios_group_size (handle, 4, total_size);
    adios_write (handle, "NX", &NX);
    adios_write (handle, "temperature", t); 
    adios_close (handle);
    adios_finalize (rank);
    MPI_Finalize(); 
    return 0;
}
\end{lstlisting}
